\documentclass[11pt]{article}
\usepackage[margin=1in]{geometry}
\usepackage{amsmath}
\usepackage{amsfonts}
\usepackage{graphicx}
\usepackage{tikz}
\usepackage[american]{circuitikz}
\usepackage{enumitem}
\usepackage{fancyhdr}
\usepackage{pgfplots}
\pgfplotsset{compat=1.18} 

\pagestyle{fancy}
\fancyhf{}
\lhead{ECEN 222 - Electronic Circuits}
\rhead{Spring 2026}
\cfoot{\thepage}

\begin{document}

\begin{center}
    {\LARGE \textbf{Assignment 01}}\\
    {\LARGE \textbf{Time Domain Analysis of RLC Circuits}}\\
    \vspace{0.3cm}
    {\large ECEN 222, Spring 2026}\\
    \vspace{0.2cm}
    {\large University of Nebraska-Lincoln}\\
    \vspace{0.5cm}
\end{center}

\section*{Instructions}

This assignment focuses on analyzing the time domain response of passive RLC circuits subjected to step inputs. You will sketch output voltage waveforms based on circuit topology and initial/final conditions.

\vspace{0.3cm}

\textbf{Important Notes:}
\begin{itemize}
    \item You do \textbf{not} need to solve differential equations (you may if you wish).
    \item The key is to determine:
    \begin{enumerate}
        \item The \textbf{initial condition} (at $t = 0$)
        \item The \textbf{final steady-state} (as $t \to \infty$)
        \item The \textbf{time constant} $\tau$ (for first-order) or damping characteristics (for second-order)
    \end{enumerate}
    \item For first-order circuits, sketch the waveform knowing it transitions exponentially or logarithmically between initial and final values.
    \item Label key features: initial value, final value, approximate time constant on your sketches.
\end{itemize}

\subsection*{Differential Equation General Solutions}

For those who wish to derive exact expressions, here are the standard solutions:

\vspace{0.3cm}

\textbf{First-Order Circuits:}

The general form is:
\[
x(t) = x(\infty) + [x(0) - x(\infty)]e^{-t/\tau}
\]

where $x(t)$ is the voltage or current of interest, and $\tau$ is the time constant.

\begin{itemize}
    \item \textbf{RC circuits}: $\tau = RC$
    \item \textbf{RL circuits}: $\tau = L/R$
\end{itemize}

\vspace{0.3cm}

\textbf{Second-Order RLC Circuits:}

The characteristic equation gives roots based on damping:
\[
s^2 + 2\alpha s + \omega_0^2 = 0
\]

where $\alpha = \frac{R}{2L}$ (for series RLC) and $\omega_0 = \frac{1}{\sqrt{LC}}$.

\begin{itemize}
    \item \textbf{Overdamped} ($\alpha > \omega_0$): 
    \[
    x(t) = A_1 e^{s_1 t} + A_2 e^{s_2 t}
    \]
    where $s_{1,2} = -\alpha \pm \sqrt{\alpha^2 - \omega_0^2}$ (both real)
    
    \item \textbf{Critically Damped} ($\alpha = \omega_0$):
    \[
    x(t) = (A_1 + A_2 t)e^{-\alpha t}
    \]
    
    \item \textbf{Underdamped} ($\alpha < \omega_0$):
    \[
    x(t) = e^{-\alpha t}(A_1 \cos(\omega_d t) + A_2 \sin(\omega_d t))
    \]
    where $\omega_d = \sqrt{\omega_0^2 - \alpha^2}$ is the damped natural frequency
\end{itemize}

\newpage

\section*{Problems}

\subsection*{Part A: First-Order Circuits}

For each circuit below, assume the circuit has been in steady state for $t < 0$, and the switch changes position at $t = 0$. Sketch $v_{\text{out}}(t)$ for $t > 0$. Clearly label:
\begin{itemize}
    \item Initial voltage $v_{\text{out}}(0^+)$
    \item Final voltage $v_{\text{out}}(\infty)$
    \item Time constant $\tau$ in terms of circuit elements
    \item Shape of the waveform
\end{itemize}

\begin{enumerate}

\item \textbf{Simple RC Step Response}

\noindent The switch has been open for a long time. The capacitor is initially uncharged $v_{\text{out}}(0^-) = 0$. At $t = 0$, the switch closes. Sketch $v_{\text{out}}(t)$.

\begin{center}
\begin{circuitikz}[american, scale=1.2]
    \draw (0,0) to[V, v^=$V_s$, invert] (0,3);
    \draw (0,3) -- (0.5,3);
    \draw (0.5,3) -- (1,3.3);
    \draw (1,3) -- (2,3);
    \draw (2,3) to[R, l=$R$] (4,3);
    \draw (4,3) to[C, l_=$C$, v^=$v_{\text{out}}$] (4,0);
    \draw (4,0) -- (2,0);
    \node[ground] at (2,0) {};
    \draw (2,0) -- (0,0);
    \node at (0.75,3.6) {$t=0$};
    \draw[->, thick, red] (1,3.4) -- (1,3.05);
\end{circuitikz}
\end{center}

\vspace{0.5cm}

\textbf{Sketch your answer here:}

\begin{center}
\begin{tikzpicture}
    \begin{axis}[
        width=0.7\textwidth,
        height=5cm,
        xlabel={Time},
        ylabel={$v_{\text{out}}(t)$},
        xmin=0, xmax=6,
        ymin=0, ymax=1.2,
        grid=major,
        xtick=\empty,
        ytick=\empty,
        axis lines=middle,
        thick
    ]
    \end{axis}
\end{tikzpicture}
\end{center}

\newpage

\item \textbf{RC Circuit with Initial Condition}

\noindent The capacitor has been charged to $V_0$ and the switch has been open for a long time. At $t = 0$, the switch closes. Sketch $v_{\text{out}}(t)$.

\begin{center}
\begin{circuitikz}[american, scale=1.2]
    \draw (0,3) to[C, l^=$C$, v_=$v_{\text{out}}$] (0,0);
    \draw (0,3) -- (0.5,3);
    \draw (0.5,3) -- (1,3.3);
    \draw (1,3) -- (2,3);
    \draw (2,3) to[R, l=$R$] (4,3);
    \draw (4,3) -- (4,0);
    \draw (4,0) -- (2,0);
    \node[ground] at (2,0) {};
    \draw (2,0) -- (0,0);
    \node at (0.75,3.6) {$t=0$};
    \draw[->, thick, red] (1,3.4) -- (1,3.05);
\end{circuitikz}
\end{center}

\vspace{0.5cm}

\textbf{Sketch your answer here:}

\begin{center}
\begin{tikzpicture}
    \begin{axis}[
        width=0.7\textwidth,
        height=5cm,
        xlabel={Time},
        ylabel={$v_{\text{out}}(t)$},
        xmin=0, xmax=6,
        ymin=0, ymax=1.2,
        grid=major,
        xtick=\empty,
        ytick=\empty,
        axis lines=middle,
        thick
    ]
    \end{axis}
\end{tikzpicture}
\end{center}

\newpage

\item \textbf{RL Step Response}

\noindent The switch has been open for a long time. At $t = 0$, the switch closes. Sketch $v_{\text{out}}(t)$ across the resistor.

\begin{center}
\begin{circuitikz}[american, scale=1.2]
    \draw (0,0) to[V, v^=$V_s$, invert] (0,3);
    \draw (0,3) -- (0.5,3);
    \draw (0.5,3) -- (1,3.3);
    \draw (1,3) -- (2,3);
    \draw (2,3) to[L, l=$L$] (4,3);
    \draw (4,3) to[R, l_=$R$, v^=$v_{\text{out}}$] (4,0);
    \draw (4,0) -- (2,0);
    \node[ground] at (2,0) {};
    \draw (2,0) -- (0,0);
    \node at (0.75,3.6) {$t=0$};
    \draw[->, thick, red] (1,3.4) -- (1,3.05);
\end{circuitikz}
\end{center}

\vspace{0.5cm}

\textbf{Sketch your answer here:}

\begin{center}
\begin{tikzpicture}
    \begin{axis}[
        width=0.7\textwidth,
        height=5cm,
        xlabel={Time},
        ylabel={$v_{\text{out}}(t)$},
        xmin=0, xmax=6,
        ymin=0, ymax=1.2,
        grid=major,
        xtick=\empty,
        ytick=\empty,
        axis lines=middle,
        thick
    ]
    \end{axis}
\end{tikzpicture}
\end{center}

\newpage

\item \textbf{RC Voltage Divider with Step Input}

\noindent The switch has been open for a long time. At $t = 0$, the switch closes. The capacitor is initially uncharged. Sketch $v_{\text{out}}(t)$. 

\begin{center}
\begin{circuitikz}[american, scale=1.2]
    \draw (0,0) to[V, v^=$V_s$, invert] (0,3);
    \draw (0,3) -- (0.5,3);
    \draw (0.5,3) -- (1,3.3);
    \draw (1,3) -- (2,3);
    \draw (2,3) to[R, l=$R_1$] (4,3);
    \draw (4,3) to[C, l=$C$] (6,3);
    \draw (6,3) to[R, l_=$R_2$, v^=$v_{\text{out}}$] (6,0);
    \draw (6,0) -- (3,0);
    \node[ground] at (3,0) {};
    \draw (3,0) -- (0,0);
    \node at (0.75,3.6) {$t=0$};
    \draw[->, thick, red] (1,3.4) -- (1,3.05);
\end{circuitikz}
\end{center}

\textit{Hint:} Consider the voltage division at $t = 0^+$ and $t = \infty$. The time constant is $\tau = (R_1 + R_2)C$.

\vspace{0.5cm}

\textbf{Sketch your answer here:}

\begin{center}
\begin{tikzpicture}
    \begin{axis}[
        width=0.7\textwidth,
        height=5cm,
        xlabel={Time},
        ylabel={$v_{\text{out}}(t)$},
        xmin=0, xmax=6,
        ymin=0, ymax=1.2,
        grid=major,
        xtick=\empty,
        ytick=\empty,
        axis lines=middle,
        thick
    ]
    \end{axis}
\end{tikzpicture}
\end{center}

\newpage

\item \textbf{RL Circuit with Parallel Resistor}

\noindent The switch has been closed for a long time. At $t = 0$, the switch opens. Sketch $v_{\text{out}}(t)$ across the inductor.

\begin{center}
\begin{circuitikz}[american, scale=1.2]
    \draw (0,0) to[V, v^=$V_s$, invert] (0,3);
    \draw (0,3) to[R, l=$R_1$] (2,3);
    \draw (2,3) -- (2.5,3);
    \draw (2.5,3) -- (3,3.3);
    \draw (3,3) -- (4,3);
    \draw (4,3) to[short] (6,3);
    \draw (6,3) to[L, l_=$L$, v^=$v_{\text{out}}$] (6,1.5);
    \draw (6,1.5) to[R, l_=$R_2$] (6,0);
    \draw (6,0) -- (3,0);
    \node[ground] at (3,0) {};
    \draw (3,0) -- (0,0);
    \node at (2.75,3.6) {$t=0$};
    \draw[->, thick, red] (3,2.9) -- (3,3.25);
\end{circuitikz}
\end{center}

\vspace{0.5cm}

\textbf{Sketch your answer here:}

\begin{center}
\begin{tikzpicture}
    \begin{axis}[
        width=0.7\textwidth,
        height=5cm,
        xlabel={Time},
        ylabel={$v_{\text{out}}(t)$},
        xmin=0, xmax=6,
        ymin=-0.2, ymax=1.2,
        grid=major,
        xtick=\empty,
        ytick=\empty,
        axis lines=middle,
        thick
    ]
    \end{axis}
\end{tikzpicture}
\end{center}

\end{enumerate}

\newpage

\subsection*{Part B: Second-Order RLC Circuits}

\noindent For the following problems, you will analyze second-order RLC circuits.

\begin{enumerate}
\setcounter{enumi}{5}

\item \textbf{Series RLC Response Classification}

\noindent Consider the series RLC circuit below. The switch has been open for a long time, and closes at $t = 0$. 

\begin{center}
\begin{circuitikz}[american, scale=1.2]
    \draw (0,0) to[V, v^=$V_s$, invert] (0,3);
    \draw (0,3) -- (0.5,3);
    \draw (0.5,3) -- (1,3.3);
    \draw (1,3) -- (2,3);
    \draw (2,3) to[R, l=$R$] (4,3);
    \draw (4,3) to[L, l=$L$] (6,3);
    \draw (6,3) to[C, l_=$C$, v^=$v_C$] (6,0);
    \draw (6,0) -- (3,0);
    \node[ground] at (3,0) {};
    \draw (3,0) -- (0,0);
    \node at (0.75,3.6) {$t=0$};
    \draw[->, thick, red] (1,3.4) -- (1,3.05);
\end{circuitikz}
\end{center}

\textbf{(a)} Sketch the three possible types of responses for $v_C(t)$:
\begin{itemize}
    \item Overdamped
    \item Critically damped
    \item Underdamped
\end{itemize}

Clearly label which is which, and indicate key characteristics (oscillation, overshoot, settling behavior).

\vspace{0.5cm}

\textbf{Sketch your answer here:}

\begin{center}
\begin{tikzpicture}
    \begin{axis}[
        width=0.7\textwidth,
        height=6cm,
        xlabel={Time},
        ylabel={$v_C(t)$},
        xmin=0, xmax=6,
        ymin=-0.2, ymax=1.5,
        grid=major,
        xtick=\empty,
        ytick=\empty,
        axis lines=middle,
        thick
    ]
    \end{axis}
\end{tikzpicture}
\end{center}

\vspace{0.5cm}

\newpage

\textbf{(b)} For each case below, determine whether the response is overdamped, critically damped, or underdamped. Use the parameters:
\[
\alpha = \frac{R}{2L}, \quad \omega_0 = \frac{1}{\sqrt{LC}}
\]

Compare $\alpha$ and $\omega_0$ to determine the damping type.

\begin{enumerate}[label=(\roman*)]
    \item $R = 100\,\Omega$, $L = 10\,$mH, $C = 10\,\mu$F
    \item $R = 20\,\Omega$, $L = 10\,$mH, $C = 10\,\mu$F
    \item $R = 632\,\Omega$, $L = 10\,$mH, $C = 10\,\mu$F
\end{enumerate}

For each case, calculate $\alpha$, $\omega_0$, and state the response type.

\newpage

\item \textbf{Parallel RLC Circuit Analysis (Extra Credit)}

\noindent Consider the parallel RLC circuit shown. The current source is a step function: $i_s(t) = I_0 \cdot u(t)$ where $u(t)$ is the unit step function.

\begin{center}
\begin{circuitikz}[american, scale=1.2]
    \draw (0,0) to[I, l=$i_s(t)$] (0,3);
    \draw (0,3) -- (2,3);
    \draw (2,3) to[R, l_=$R$] (2,0);
    \draw (2,3) -- (4,3);
    \draw (4,3) to[L, l_=$L$] (4,0);
    \draw (4,3) -- (6,3);
    \draw (6,3) to[C, l_=$C$, v^=$v_{\text{out}}$] (6,0);
    \draw (6,0) -- (3,0);
    \node[ground] at (3,0) {};
    \draw (3,0) -- (0,0);
\end{circuitikz}
\end{center}

\textbf{(a)} Without specifying values, sketch the general shape of $v_{\text{out}}(t)$ for an underdamped response. Indicate the damped oscillation frequency and the exponential envelope.

\vspace{0.5cm}

\textbf{Sketch your answer here:}

\begin{center}
\begin{tikzpicture}
    \begin{axis}[
        width=0.7\textwidth,
        height=5cm,
        xlabel={Time},
        ylabel={$v_{\text{out}}(t)$},
        xmin=0, xmax=6,
        ymin=-0.5, ymax=1.5,
        grid=major,
        xtick=\empty,
        ytick=\empty,
        axis lines=middle,
        thick
    ]
    \end{axis}
\end{tikzpicture}
\end{center}

\vspace{0.5cm}

\textbf{(b)} Choose specific values for $R$, $L$, and $C$ that would result in a critically damped response. Show your work to verify that $\alpha = \omega_0$.

For a parallel RLC: $\alpha = \frac{1}{2RC}$ and $\omega_0 = \frac{1}{\sqrt{LC}}$.

\end{enumerate}

\newpage

\section*{Submission Guidelines}

\begin{itemize}
    \item Show all work and reasoning clearly.
    \item Sketches should be neat and properly labeled.
    \item For sketches, indicate time axis (you may use multiples of $\tau$ for first-order).
    \item Clearly mark initial conditions, final values, and time constants.
\end{itemize}



\end{document}
