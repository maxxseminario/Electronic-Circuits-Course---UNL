\documentclass[11pt]{article}
\usepackage{amsmath}
\usepackage{amssymb}
\usepackage{graphicx}
\usepackage{listings}
\usepackage{xcolor}
\usepackage{geometry}
\usepackage{fancyhdr}
\usepackage{tikz}
\usepackage{circuitikz}
\usetikzlibrary{shapes,arrows,positioning,calc}

\geometry{margin=1in}
\pagestyle{fancy}
\fancyhf{}
\rhead{Parametric Resonator Design}
\lhead{DSP HW2  - maxxseminario}
\cfoot{\thepage}

% MATLAB code styling
\lstset{
    language=Matlab,
    basicstyle=\ttfamily\small,
    keywordstyle=\color{blue},
    commentstyle=\color{green!60!black},
    stringstyle=\color{red},
    numbers=left,
    numberstyle=\tiny\color{gray},
    stepnumber=1,
    numbersep=5pt,
    backgroundcolor=\color{gray!10},
    frame=single,
    rulecolor=\color{black},
    breaklines=true,
    breakatwhitespace=true,
    tabsize=2
}

% Define block diagram styles
\tikzstyle{block} = [draw, fill=blue!20, rectangle, minimum height=3em, minimum width=6em]
\tikzstyle{sum} = [draw, fill=blue!20, circle, node distance=1cm]
\tikzstyle{input} = [coordinate]
\tikzstyle{output} = [coordinate]
\tikzstyle{pinstyle} = [pin edge={to-,thin,black}]

\title{DSP HW2 Part 2: Digital Parametric Resonators}
\author{Maxx Seminaio}
\date{October 12, 2025}

\begin{document}

\maketitle

\section{Introduction}

A parametric resonator is a second-order digital filter designed to produce sustained oscillations at a specific frequency when excited by a brief input signal. These filters are used as blocks in digital audio synthesis and can simulate the resonant behavior of physical instruments.



% \begin{figure}[h!]
% \centering
% \begin{tikzpicture}[auto, node distance=2cm,>=latex']
%     % Input and output nodes
%     \node [input, name=input] {};
%     \node [block, right of=input] (gain) {$G$};
%     \node [sum, right of=gain, node distance=3cm] (sum1) {$+$};
%     \node [block, right of=sum1, node distance=3cm] (delay1) {$z^{-1}$};
%     \node [sum, below of=delay1] (sum2) {$+$};
%     \node [block, below of=sum2] (delay2) {$z^{-1}$};
%     \node [output, right of=delay1, node distance=3cm] (output) {};
    
%     % Feedback paths
%     \node [block, left of=sum2, node distance=3cm] (a1) {$-a_1$};
%     \node [block, below of=a1] (a2) {$-a_2$};
    
%     % Draw connections
%     \draw [->] (input) -- node {$x[n]$} (gain);
%     \draw [->] (gain) -- (sum1);
%     \draw [->] (sum1) -- (delay1);
%     \draw [->] (delay1) -- node [name=y] {$y[n]$} (output);
%     \draw [->] (delay1) -- (sum2);
%     \draw [->] (sum2) -- (delay2);
    
%     % Feedback connections
%     \draw [->] (delay1) -- ++(0,-1) -| (a1);
%     \draw [->] (a1) -- (sum1);
%     \draw [->] (delay2) -- ++(-2,0) |- (a2);
%     \draw [->] (a2) -- ++(1.5,0) |- (sum1);
    
%     % Labels
%     \node [above of=sum1, node distance=0.7cm] {$w[n]$};
%     \node [right of=delay1, node distance=1.5cm] {$w[n-1]$};
%     \node [right of=delay2, node distance=1.5cm] {$w[n-2]$};
    
% \end{tikzpicture}
% \caption{Direct Form II implementation of second-order resonator}
% \label{fig:direct_form_ii}
% \end{figure}

\section{Mathematical Derivation}

\subsection{Time Domain Difference Equation}

A second-order digital resonator can be described by the following difference equation:

\begin{equation}
y[n] = b_0 x[n] - a_1 y[n-1] - a_2 y[n-2]
\end{equation}

where:
\begin{itemize}
    \item $x[n]$ is the input signal
    \item $y[n]$ is the output signal
    \item $b_0$ is the feedforward coefficient
    \item $a_1, a_2$ are the feedback coefficients
\end{itemize}


The difference equation $y[n] = Gx[n] - a_1 y[n-1] - a_2 y[n-2]$ creates oscillations because it implements a \textbf{digital memory system with delayed feedback}. When you input a brief pulse, the system ``remembers'' its recent outputs through the $y[n-1]$ and $y[n-2]$ terms, and the carefully chosen coefficients $a_1$ and $a_2$ cause these delayed versions to reinforce each other in a periodic pattern that naturally 'rings' at the designed frequency. 

\subsection{Z-Transform and Transfer Function}

Taking the z-transform of equation (1):

\begin{equation}
Y(z) = b_0 X(z) - a_1 z^{-1} Y(z) - a_2 z^{-2} Y(z)
\end{equation}

Rearranging to solve for the transfer function:

\begin{equation}
Y(z)[1 + a_1 z^{-1} + a_2 z^{-2}] = b_0 X(z)
\end{equation}

\begin{equation}
H(z) = \frac{Y(z)}{X(z)} = \frac{b_0}{1 + a_1 z^{-1} + a_2 z^{-2}}
\end{equation}

\subsection{Pole Placement for Resonance}

To create a resonator at frequency $\omega_0$ with decay factor $R$, we place complex conjugate poles at:

\begin{equation}
p_{1,2} = R e^{\pm j\omega_0}
\end{equation}

The denominator polynomial with these poles is:

\begin{align}
D(z) &= (1 - R e^{j\omega_0} z^{-1})(1 - R e^{-j\omega_0} z^{-1}) \\
&= 1 - R(e^{j\omega_0} + e^{-j\omega_0})z^{-1} + R^2 z^{-2} \\
&= 1 - 2R\cos(\omega_0)z^{-1} + R^2 z^{-2}
\end{align}

Therefore, the coefficients are:
\begin{align}
a_1 &= -2R\cos(\omega_0) \\
a_2 &= R^2
\end{align}

\subsection{Frequency Domain Analysis}

The frequency response is obtained by substituting $z = e^{j\omega}$:

\begin{equation}
H(e^{j\omega}) = \frac{b_0}{1 + a_1 e^{-j\omega} + a_2 e^{-j2\omega}}
\end{equation}

Substituting our coefficients:

\begin{equation}
H(e^{j\omega}) = \frac{b_0}{1 - 2R\cos(\omega_0)e^{-j\omega} + R^2 e^{-j2\omega}}
\end{equation}

\subsection{Gain Normalization}

To ensure unity gain at the resonant frequency $\omega_0$, we set:

\begin{equation}
|H(e^{j\omega_0})| = 1
\end{equation}

The magnitude at resonance is:

\begin{equation}
|H(e^{j\omega_0})| = \frac{b_0}{|1 - 2R\cos(\omega_0)e^{-j\omega_0} + R^2 e^{-j2\omega_0}|}
\end{equation}

After algebraic manipulation, the normalization factor is:

\begin{equation}
b_0 = G = (1-R)\sqrt{1 - 2R\cos(2\omega_0) + R^2}
\end{equation}

\subsection{Digital Frequency Conversion}

For a desired analog frequency $f$ Hz and sampling rate $F_s$:

\begin{equation}
\omega_0 = \frac{2\pi f}{F_s}
\end{equation}

\section{MATLAB Implementation}

\subsection{Resonator Coefficient Function}

\begin{lstlisting}[caption=res\_coeffs.m - Resonator coefficient generator]
function [a, b] = res_coeffs(freq, R, Fs)
    % RES_COEFFS Generate coefficients for parametric resonator
    %
    % Inputs:
    %   freq - Desired resonant frequency in Hz
    %   R    - Pole radius (0 < R < 1), controls decay rate
    %   Fs   - Sampling frequency in Hz
    %
    % Outputs:
    %   a - Denominator coefficients [1, a1, a2]
    %   b - Numerator coefficient [G]
    
    % Convert analog frequency to digital frequency
    omega_0 = (2 * pi * freq) / Fs;
    
    % Calculate denominator coefficients (feedback)
    a1 = -2 * R * cos(omega_0);
    a2 = R^2;
    
    % Calculate gain normalization factor
    G = (1 - R) * sqrt(1 - 2*R*cos(2*omega_0) + R^2);
    
    % Return filter coefficients
    b = [G];           % Numerator (feedforward)
    a = [1, a1, a2];   % Denominator (feedback)
end
\end{lstlisting}



\section{Example Figures}


\begin{figure*}[htb]
\centering
\includegraphics[width=\textwidth]{Resonance_Sweep.png}
\caption{Resonance Sweep in time domain.}
\label{fig:r_sweep}
\end{figure*}


\begin{figure*}[htb]
\centering
\includegraphics[width=\textwidth]{Freqz_RSweep.png}
\caption{Resonance Sweep in frequency domain.}
\label{fig:r_sweep_freq}
\end{figure*}


\begin{figure*}[htb]
\centering
\includegraphics[width=\textwidth]{Frequency_Sweep.png}
\caption{Resonance Frequency Sweep in time domain.}
\label{fig:freq_sweep}
\end{figure*}


\begin{figure*}[htb]
\centering
\includegraphics[width=\textwidth]{Freqz_FSweep.png}
\caption{Resonance Frequency Sweep in frequency domain.}
\label{fig:freq_sweep_freq}
\end{figure*}












% \subsection{Complete Resonator Example}

% \begin{lstlisting}[caption=resonator\_example.m - Complete implementation example]
% %% Parametric Resonator Example
% clear; clc; close all;

% % Parameters
% Fs = 8000;          % Sampling frequency (Hz)
% freq = 440;         % Resonant frequency (A4 note)
% R = 0.99;           % Pole radius (close to 1 for sharp resonance)
% duration = 2;       % Duration in seconds

% % Generate resonator coefficients
% [a, b] = res_coeffs(freq, R, Fs);

% % Create impulse input
% N = duration * Fs;
% impulse = [1, zeros(1, N-1)];

% % Generate impulse response
% y = filter(b, a, impulse);

% % Plot impulse response
% figure(1);
% subplot(2,1,1);
% t = (0:length(y)-1) / Fs;
% plot(t(1:1000), y(1:1000), 'LineWidth', 1.5);
% xlabel('Time (seconds)');
% ylabel('Amplitude');
% title('Impulse Response of 440 Hz Resonator');
% grid on;

% % Plot frequency response
% subplot(2,1,2);
% [H, w] = freqz(b, a, 1024);
% plot(w*Fs/(2*pi), 20*log10(abs(H)), 'LineWidth', 1.5);
% xlabel('Frequency (Hz)');
% ylabel('Magnitude (dB)');
% title('Frequency Response');
% grid on;
% xlim([0, Fs/2]);

% % Play the sound
% soundsc(y, Fs);

% %% Parameter Study
% figure(2);

% % Different R values
% R_values = [0.9, 0.95, 0.99];
% colors = ['r', 'g', 'b'];

% for i = 1:length(R_values)
%     [a_temp, b_temp] = res_coeffs(freq, R_values(i), Fs);
%     y_temp = filter(b_temp, a_temp, impulse);
    
%     subplot(2,1,1);
%     plot(t(1:500), y_temp(1:500), colors(i), 'LineWidth', 1.5);
%     hold on;
    
%     subplot(2,1,2);
%     [H_temp, w] = freqz(b_temp, a_temp, 1024);
%     plot(w*Fs/(2*pi), abs(H_temp), colors(i), 'LineWidth', 1.5);
%     hold on;
% end

% subplot(2,1,1);
% xlabel('Time (seconds)');
% ylabel('Amplitude');
% title('Impulse Response for Different R Values');
% legend('R=0.9', 'R=0.95', 'R=0.99');
% grid on;

% subplot(2,1,2);
% xlabel('Frequency (Hz)');
% ylabel('Magnitude');
% title('Frequency Response for Different R Values');
% legend('R=0.9', 'R=0.95', 'R=0.99');
% grid on;
% xlim([200, 800]);
% \end{lstlisting}

% \section{Parameter Effects}

% \subsection{Pole Radius (R)}
% \begin{itemize}
%     \item \textbf{R close to 0}: Fast decay, broad frequency response
%     \item \textbf{R close to 1}: Slow decay, sharp frequency response
%     \item \textbf{R = 1}: Marginal stability, pure oscillation
%     \item \textbf{R > 1}: Unstable system (avoid!)
% \end{itemize}

% \subsection{Resonant Frequency}
% \begin{itemize}
%     \item Determines the angle of poles in z-plane: $\pm\omega_0$
%     \item Must satisfy Nyquist criterion: $f < F_s/2$
%     \item Higher frequencies place poles at larger angles from real axis
% \end{itemize}

% \section{Applications and Extensions}

% \subsection{Musical Synthesis}
% Multiple resonators can be combined to create:
% \begin{itemize}
%     \item Harmonic series for realistic instrument timbres
%     \item Chord progressions by summing different frequencies
%     \item Percussive sounds with varying decay rates
% \end{itemize}

% \subsection{Advanced Techniques}
% \begin{itemize}
%     \item \textbf{Time-varying parameters}: Modulate R or frequency over time
%     \item \textbf{Parallel resonator banks}: Multiple resonators for complex spectra
%     \item \textbf{Feedback networks}: Connecting resonators for richer sounds
% \end{itemize}

% \section{Conclusion}

% Parametric resonators provide a powerful and computationally efficient method for digital audio synthesis. By understanding the mathematical foundation and parameter relationships, you can create a wide variety of synthetic sounds ranging from simple tones to complex musical textures.

% The key insight is that placing poles near the unit circle in the z-plane creates sustained oscillations, while the pole radius R controls the decay rate and frequency selectivity. This simple second-order structure forms the basis for many advanced digital audio processing techniques.

\end{document}