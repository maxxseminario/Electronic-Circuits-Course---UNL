
\documentclass[10pt, aspectratio=169, handout]{beamer}
\usefonttheme{professionalfonts}

\mode<presentation>
{
  \usetheme{Berkeley}
  \usecolortheme{beaver}
  \usefonttheme{default}
  \setbeamertemplate{navigation symbols}{}
  \setbeamertemplate{caption}[numbered]
} 

\setbeamertemplate{footline}{%
  \leavevmode%
  \hbox{%
    \begin{beamercolorbox}[wd=.85\paperwidth,ht=2.5ex,dp=1ex,left]{author in head/foot}%
      \usebeamerfont{author in head/foot}Digital Signal Processing, Fall 2025%
    \end{beamercolorbox}%
    \begin{beamercolorbox}[wd=.15\paperwidth,ht=2.5ex,dp=1ex,right]{date in head/foot}%
      \hspace*{0.5em}\insertframenumber{} / \inserttotalframenumber\hspace*{0.5em}%
    \end{beamercolorbox}%
  }%
  \vskip0pt%
}

\usepackage[english]{babel}
\usepackage[utf8x]{inputenc}
\usepackage{tikz}
\usepackage{pgfplots}
\usepackage{array}
\usepackage{makecell}
\usepackage{verbatim}
\usepackage{graphicx}
\usepackage{subcaption}
\usepackage{amsfonts}
\usepackage{amsmath}
\usepackage{bm}
\usepackage{epstopdf}
\captionsetup{compatibility=false}

\usetikzlibrary{calc}
\usetikzlibrary{pgfplots.fillbetween, backgrounds}
\usetikzlibrary{positioning}
\usetikzlibrary{pgfplots.groupplots}
\usetikzlibrary{plotmarks}
\usetikzlibrary{calc}

\usepgfplotslibrary{groupplots}
\pgfplotsset{compat=newest} 

\usepackage{ifthen}
\newboolean{showresults}
\setboolean{showresults}{false}


\usepackage{hyperref}
\hypersetup{
    colorlinks=true,
    linkcolor=blue,
    filecolor=magenta,      
    urlcolor=cyan,
}

\title[ECEN 463/863]{The Discrete Fourier Transform}
\author{Maxx Seminario}
\institute{University of Nebraska-Lincoln}
\date{October 3, 2025}

\begin{document}
\begin{frame}
  \titlepage
\end{frame}

\section{Introduction}

\begin{frame}{Overview: Representation of Sequences}
\begin{itemize}
    \item \textbf{Motivation}: How to represent arbitrary input sequences in frequency domain?
    \begin{itemize}
        \item We know LTI system response to $e^{j\omega n}$ is $H(e^{j\omega})e^{j\omega n}$
        \item Need to decompose arbitrary signals into complex exponentials
    \end{itemize}
    
    \item \textbf{Key Questions}:
    \begin{itemize}
        \item Can we represent any sequence as a sum of complex exponentials?
        \item What are the conditions for such representations to exist?
        \item How do we compute the representation?
    \end{itemize}
    
    \item \textbf{Today's Topics}:
    \begin{itemize}
        \item Fourier Transform pair for discrete-time signals
        \item Convergence conditions
        \item Symmetry properties
        \item Fourier Transform theorems
    \end{itemize}
\end{itemize}
\end{frame}

\section{The Fourier Transform Pair}

\begin{frame}{The Discrete-Time Fourier Transform (DTFT)}
\textbf{Fourier Transform Pair}:
\begin{align}
    x[n] &= \frac{1}{2\pi} \int_{-\pi}^{\pi} X(e^{j\omega})e^{j\omega n}d\omega \quad \text{(Synthesis)} \\
    X(e^{j\omega}) &= \sum_{n=-\infty}^{\infty} x[n]e^{-j\omega n} \quad \text{(Analysis)}
\end{align}

% \vspace{0.3cm}
\textbf{Key Interpretations}:
\begin{itemize}
    \item \textbf{Synthesis}: Represents $x[n]$ as superposition of complex sinusoids
    \item \textbf{Analysis}: Determines "how much" of each frequency is in $x[n]$
    \item $X(e^{j\omega})$ is generally complex-valued
    \item Integration can be over any interval of length $2\pi$
\end{itemize}

% \vspace{0.3cm}
\textbf{Connection to LTI Systems}:
\[
    H(e^{j\omega}) = \sum_{n=-\infty}^{\infty} h[n]e^{-j\omega n}
\]
Frequency response = DTFT of impulse response
\end{frame}

\begin{frame}{Complex Representation of DTFT}
\textbf{Rectangular Form}:
\[
    X(e^{j\omega}) = X_R(e^{j\omega}) + jX_I(e^{j\omega})
\]

% \vspace{0.3cm}
\textbf{Polar Form}:
\[
    X(e^{j\omega}) = |X(e^{j\omega})|e^{j\angle X(e^{j\omega})}
\]

% \vspace{0.3cm}
\textbf{Phase Considerations}:
\begin{itemize}
    \item Phase is not unique (can add multiples of $2\pi$)
    \item \textbf{Principal value}: $\text{ARG}[X(e^{j\omega})] \in [-\pi, \pi]$
    \item \textbf{Continuous phase}: $\arg[X(e^{j\omega})]$ (unwrapped phase)
\end{itemize}

% \vspace{0.3cm}
\textbf{Periodicity}:
\begin{itemize}
    \item $X(e^{j\omega})$ is periodic in $\omega$ with period $2\pi$
    \item Similar to Fourier series for periodic functions
    \item Eq. (2.131) is Fourier series of periodic function $X(e^{j\omega})$
\end{itemize}
\end{frame}

\section{Convergence Conditions}

\begin{frame}{Convergence of the DTFT}
\textbf{Question}: When does $X(e^{j\omega}) = \sum_{n=-\infty}^{\infty} x[n]e^{-j\omega n}$ converge?

% \vspace{0.3cm}
\textbf{Absolute Summability (Sufficient Condition)}:
\begin{align}
    |X(e^{j\omega})| &= \left|\sum_{n=-\infty}^{\infty} x[n]e^{-j\omega n}\right| \\
    &\leq \sum_{n=-\infty}^{\infty} |x[n]||e^{-j\omega n}| \\
    &= \sum_{n=-\infty}^{\infty} |x[n]| < \infty
\end{align}

% \vspace{0.3cm}
\textbf{Implications}:
\begin{itemize}
    \item If $\sum_{n=-\infty}^{\infty} |x[n]| < \infty$, then $X(e^{j\omega})$ exists
    \item Series converges uniformly to a continuous function
    \item All stable sequences have Fourier transforms
    \item All FIR systems have finite, continuous frequency responses
\end{itemize}
\end{frame}

\begin{frame}{Example: Suddenly-Applied Exponential}
\textbf{Sequence}: $x[n] = a^n u[n]$

% \vspace{0.3cm}
\textbf{DTFT Calculation}:
\begin{align}
    X(e^{j\omega}) &= \sum_{n=0}^{\infty} a^n e^{-j\omega n} \\
    &= \sum_{n=0}^{\infty} (ae^{-j\omega})^n \\
    &= \frac{1}{1 - ae^{-j\omega}} \quad \text{if } |ae^{-j\omega}| < 1
\end{align}

% \vspace{0.3cm}
\textbf{Convergence Condition}: $|a| < 1$

% \vspace{0.3cm}
\textbf{Absolute Summability Check}:
\[
    \sum_{n=0}^{\infty} |a|^n = \frac{1}{1 - |a|} < \infty \quad \text{if } |a| < 1
\]

This confirms that absolute summability guarantees convergence.
\end{frame}

\begin{frame}{Square Summability}
\textbf{Alternative Condition}: Some sequences are not absolutely summable but are square summable:
\[
    \sum_{n=-\infty}^{\infty} |x[n]|^2 < \infty
\]

% \vspace{0.3cm}
\textbf{Mean-Square Convergence}:
\[
    \lim_{M \to \infty} \int_{-\pi}^{\pi} |X(e^{j\omega}) - X_M(e^{j\omega})|^2 d\omega = 0
\]
where $X_M(e^{j\omega}) = \sum_{n=-M}^{M} x[n]e^{-j\omega n}$

% \vspace{0.3cm}
\textbf{Important Distinction}:
\begin{itemize}
    \item Error may not approach zero at each $\omega$
    \item Total "energy" in error approaches zero
    \item Leads to Gibbs phenomenon at discontinuities
\end{itemize}
\end{frame}

\begin{frame}{Example: Ideal Lowpass Filter}
\textbf{Frequency Response}:
\[
    H_{lp}(e^{j\omega}) = \begin{cases}
        1, & |\omega| < \omega_c \\
        0, & \omega_c < |\omega| \leq \pi
    \end{cases}
\]

% \vspace{0.3cm}
\textbf{Impulse Response}:
\begin{align}
    h_{lp}[n] &= \frac{1}{2\pi} \int_{-\omega_c}^{\omega_c} e^{j\omega n} d\omega \\
    &= \frac{1}{2\pi jn} [e^{j\omega_c n} - e^{-j\omega_c n}] \\
    &= \frac{\sin(\omega_c n)}{\pi n}, \quad -\infty < n < \infty
\end{align}

% \vspace{0.3cm}
\textbf{Properties}:
\begin{itemize}
    \item \textbf{Noncausal}: $h_{lp}[n] \neq 0$ for $n < 0$
    \item \textbf{Not absolutely summable}: Decays as $1/n$
    \item \textbf{Square summable}: Mean-square convergence
\end{itemize}
\end{frame}




\begin{frame}{Gibbs Phenomenon}
\textbf{Finite Sum Approximation}:
\[
    H_M(e^{j\omega}) = \sum_{n=-M}^{M} \frac{\sin(\omega_c n)}{\pi n} e^{-j\omega n}
\]

% \vspace{0.3cm}
% \textbf{Gibbs Phenomenon Illustration}:
% \begin{itemize}
%     \item As $M$ increases, oscillations near $\omega = \pm\omega_c$ become more rapid
%     \item Maximum overshoot remains approximately 9\%
%     \item Oscillations converge in location toward discontinuities
% \end{itemize}

% \vspace{0.3cm}
\begin{center}
\includegraphics[width=0.4\textwidth]{gibbs_phenom.png}
\end{center}
% Note: Replace with actual figure or use simpler TikZ code

\textbf{Key Points}:
\begin{itemize}
    \item Oscillations near discontinuity persist as $M \to \infty$
    \item Maximum overshoot $\approx 9\%$ (doesn't decrease)
    \item Oscillations become more rapid but localized
\end{itemize}
\end{frame}





\section{Special Cases}

\begin{frame}{Fourier Transform of Special Sequences}
\textbf{Case 1: Constant Sequence}
\begin{itemize}
    \item Sequence: $x[n] = 1$ for all $n$
    \item Neither absolutely nor square summable
    \item DTFT (using impulse functions):
    \[
        X(e^{j\omega}) = \sum_{r=-\infty}^{\infty} 2\pi\delta(\omega + 2\pi r)
    \]
\end{itemize}

% \vspace{0.3cm}
\textbf{Case 2: Complex Exponential}
\begin{itemize}
    \item Sequence: $x[n] = e^{j\omega_0 n}$, $-\pi < \omega_0 \leq \pi$
    \item DTFT:
    \[
        X(e^{j\omega}) = \sum_{r=-\infty}^{\infty} 2\pi\delta(\omega - \omega_0 + 2\pi r)
    \]
    \item Verification: Substituting into synthesis equation yields $x[n] = e^{j\omega_0 n}$
\end{itemize}
\end{frame}

\begin{frame}{Unit Step Sequence}
\textbf{Sequence}: $u[n]$

% \vspace{0.3cm}
\textbf{DTFT}:
\[
    U(e^{j\omega}) = \frac{1}{1 - e^{-j\omega}} + \sum_{r=-\infty}^{\infty} \pi\delta(\omega + 2\pi r)
\]

% \vspace{0.3cm}
\textbf{Interpretation}:
\begin{itemize}
    \item First term: Response for $\omega \neq 0$
    \item Second term: DC component (impulses at $\omega = 2\pi r$)
    \item Neither absolutely nor square summable
    \item Requires generalized function theory
\end{itemize}

% \vspace{0.3cm}
\textbf{General Form}: For sequences with discrete frequency components:
\[
    x[n] = \sum_k a_k e^{j\omega_k n} \Rightarrow X(e^{j\omega}) = \sum_{r=-\infty}^{\infty} \sum_k 2\pi a_k \delta(\omega - \omega_k + 2\pi r)
\]
\end{frame}

\section{Symmetry Properties}

\begin{frame}{Conjugate Symmetry Definitions}
\textbf{For Sequences}:
\begin{itemize}
    \item \textbf{Conjugate-symmetric}: $x_e[n] = x_e^*[-n]$
    \item \textbf{Conjugate-antisymmetric}: $x_o[n] = -x_o^*[-n]$
\end{itemize}

% \vspace{0.3cm}
\textbf{Decomposition}: Any sequence can be written as:
\begin{align}
    x[n] &= x_e[n] + x_o[n] \\
    x_e[n] &= \frac{1}{2}(x[n] + x^*[-n]) \\
    x_o[n] &= \frac{1}{2}(x[n] - x^*[-n])
\end{align}

% \vspace{0.3cm}
\textbf{For Real Sequences}:
\begin{itemize}
    \item \textbf{Even}: $x_e[n] = x_e[-n]$ (conjugate-symmetric when real)
    \item \textbf{Odd}: $x_o[n] = -x_o[-n]$ (conjugate-antisymmetric when real)
\end{itemize}
\end{frame}

\begin{frame}{Symmetry Properties of DTFT}
\textbf{General Complex Sequences}:
\begin{center}
\begin{tabular}{|l|l|}
\hline
\textbf{Sequence} & \textbf{Fourier Transform} \\
\hline
$x^*[n]$ & $X^*(e^{-j\omega})$ \\
$x^*[-n]$ & $X^*(e^{j\omega})$ \\
$\text{Re}\{x[n]\}$ & $X_e(e^{j\omega})$ (conjugate-symmetric part) \\
$j\text{Im}\{x[n]\}$ & $X_o(e^{j\omega})$ (conjugate-antisymmetric part) \\
$x_e[n]$ & $X_R(e^{j\omega}) = \text{Re}\{X(e^{j\omega})\}$ \\
$x_o[n]$ & $jX_I(e^{j\omega}) = j\text{Im}\{X(e^{j\omega})\}$ \\
\hline
\end{tabular}
\end{center}

% \vspace{0.3cm}
\textbf{Key Insight}: 
\begin{itemize}
    \item Conjugate-symmetric sequences $\leftrightarrow$ Real transforms
    \item Conjugate-antisymmetric sequences $\leftrightarrow$ Imaginary transforms
\end{itemize}
\end{frame}

\begin{frame}{Symmetry Properties for Real Sequences}
\textbf{If $x[n]$ is real}:
\begin{center}
\begin{tabular}{|l|l|}
\hline
\textbf{Property} & \textbf{Result} \\
\hline
DTFT & $X(e^{j\omega}) = X^*(e^{-j\omega})$ (conjugate symmetric) \\
Real part & $X_R(e^{j\omega}) = X_R(e^{-j\omega})$ (even function) \\
Imaginary part & $X_I(e^{j\omega}) = -X_I(e^{-j\omega})$ (odd function) \\
Magnitude & $|X(e^{j\omega})| = |X(e^{-j\omega})|$ (even function) \\
Phase & $\angle X(e^{j\omega}) = -\angle X(e^{-j\omega})$ (odd function) \\
Even part of $x[n]$ & Transforms to $X_R(e^{j\omega})$ \\
Odd part of $x[n]$ & Transforms to $jX_I(e^{j\omega})$ \\
\hline
\end{tabular}
\end{center}

% \vspace{0.3cm}
\textbf{Practical Implication}: For real signals, we only need to compute DTFT for $\omega \in [0, \pi]$
\end{frame}



\begin{frame}{Example: Symmetry Properties}
\textbf{Sequence}: $x[n] = a^n u[n]$, $|a| < 1$ (real)

% \vspace{0.3cm}
\textbf{DTFT}: $X(e^{j\omega}) = \frac{1}{1 - ae^{-j\omega}}$

% \vspace{0.3cm}
\textbf{Components}:
\begin{align}
    X_R(e^{j\omega}) &= \frac{1 - a\cos\omega}{1 + a^2 - 2a\cos\omega} \quad \text{(even)} \\
    X_I(e^{j\omega}) &= \frac{-a\sin\omega}{1 + a^2 - 2a\cos\omega} \quad \text{(odd)} \\
    |X(e^{j\omega})| &= \frac{1}{(1 + a^2 - 2a\cos\omega)^{1/2}} \quad \text{(even)} \\
    \angle X(e^{j\omega}) &= \tan^{-1}\left(\frac{-a\sin\omega}{1 - a\cos\omega}\right) \quad \text{(odd)}
\end{align}

% \vspace{0.3cm}
\textbf{Verification}: Each property satisfies the symmetry conditions for real sequences
\end{frame}

\section{Fourier Transform Theorems}

\begin{frame}{Overview of DTFT Theorems}
\textbf{Key Theorems}:
\begin{enumerate}
    \item \textbf{Linearity}: $ax[n] + by[n] \leftrightarrow aX(e^{j\omega}) + bY(e^{j\omega})$
    \item \textbf{Time Shifting}: $x[n-n_d] \leftrightarrow e^{-j\omega n_d}X(e^{j\omega})$
    \item \textbf{Frequency Shifting}: $e^{j\omega_0 n}x[n] \leftrightarrow X(e^{j(\omega-\omega_0)})$
    \item \textbf{Time Reversal}: $x[-n] \leftrightarrow X(e^{-j\omega})$
    \item \textbf{Differentiation}: $nx[n] \leftrightarrow j\frac{dX(e^{j\omega})}{d\omega}$
    \item \textbf{Convolution}: $x[n] * y[n] \leftrightarrow X(e^{j\omega})Y(e^{j\omega})$
    \item \textbf{Multiplication}: $x[n]y[n] \leftrightarrow \frac{1}{2\pi}\int_{-\pi}^{\pi} X(e^{j\theta})Y(e^{j(\omega-\theta)})d\theta$
\end{enumerate}

% \vspace{0.3cm}
\textbf{Why Important?}
\begin{itemize}
    \item Simplify analysis of complex signals
    \item Transform difficult operations into simple ones
    \item Foundation for filter design and signal processing
\end{itemize}
\end{frame}

\begin{frame}{Linearity and Time Shifting}
\textbf{1. Linearity}:
\[
    ax_1[n] + bx_2[n] \overset{\mathcal{F}}{\longleftrightarrow} aX_1(e^{j\omega}) + bX_2(e^{j\omega})
\]

% \vspace{0.3cm}
\textbf{2. Time Shifting}:
\[
    x[n - n_d] \overset{\mathcal{F}}{\longleftrightarrow} e^{-j\omega n_d}X(e^{j\omega})
\]

% \vspace{0.3cm}
\textbf{Proof of Time Shifting}:
\begin{align}
    \mathcal{F}\{x[n-n_d]\} &= \sum_{n=-\infty}^{\infty} x[n-n_d]e^{-j\omega n} \\
    &= \sum_{m=-\infty}^{\infty} x[m]e^{-j\omega(m+n_d)} \quad (m = n-n_d) \\
    &= e^{-j\omega n_d} \sum_{m=-\infty}^{\infty} x[m]e^{-j\omega m} \\
    &= e^{-j\omega n_d}X(e^{j\omega})
\end{align}
\end{frame}

\begin{frame}{Frequency Shifting and Time Reversal}
\textbf{3. Frequency Shifting (Modulation)}:
\[
    e^{j\omega_0 n}x[n] \overset{\mathcal{F}}{\longleftrightarrow} X(e^{j(\omega-\omega_0)})
\]

\textbf{Application}: Modulation shifts spectrum by $\omega_0$

% \vspace{0.5cm}
\textbf{4. Time Reversal}:
\[
    x[-n] \overset{\mathcal{F}}{\longleftrightarrow} X(e^{-j\omega})
\]

For real sequences: $x[-n] \overset{\mathcal{F}}{\longleftrightarrow} X^*(e^{j\omega})$

% \vspace{0.3cm}
\textbf{Example Application}:
\begin{itemize}
    \item Signal $x[n] = \cos(\omega_0 n) = \frac{1}{2}(e^{j\omega_0 n} + e^{-j\omega_0 n})$
    \item Using frequency shifting: spectrum has impulses at $\pm\omega_0$
\end{itemize}
\end{frame}

\begin{frame}{Differentiation in Frequency}
\textbf{5. Differentiation Theorem}:
\[
    nx[n] \overset{\mathcal{F}}{\longleftrightarrow} j\frac{dX(e^{j\omega})}{d\omega}
\]

% \vspace{0.3cm}
\textbf{Proof Sketch}:
\begin{align}
    \frac{dX(e^{j\omega})}{d\omega} &= \frac{d}{d\omega}\sum_{n=-\infty}^{\infty} x[n]e^{-j\omega n} \\
    &= \sum_{n=-\infty}^{\infty} x[n](-jn)e^{-j\omega n} \\
    &= -j\sum_{n=-\infty}^{\infty} nx[n]e^{-j\omega n}
\end{align}

% \vspace{0.3cm}
\textbf{Applications}:
\begin{itemize}
    \item Finding moments of sequences
    \item Analyzing group delay of filters
    \item Computing derivatives of frequency response
\end{itemize}
\end{frame}

\begin{frame}{Parseval's Theorem}
\textbf{Energy Conservation}:
\[
    E = \sum_{n=-\infty}^{\infty} |x[n]|^2 = \frac{1}{2\pi} \int_{-\pi}^{\pi} |X(e^{j\omega})|^2 d\omega
\]

% \vspace{0.3cm}
\textbf{Interpretation}:
\begin{itemize}
    \item Total energy in time domain = Total energy in frequency domain
    \item $|X(e^{j\omega})|^2$ is the \textbf{energy density spectrum}
    \item Shows how energy is distributed across frequencies
\end{itemize}

% \vspace{0.3cm}
\textbf{General Form}:
\[
    \sum_{n=-\infty}^{\infty} x[n]y^*[n] = \frac{1}{2\pi} \int_{-\pi}^{\pi} X(e^{j\omega})Y^*(e^{j\omega}) d\omega
\]

% \vspace{0.3cm}
\textbf{Application}: Signal power calculation, filter energy analysis
\end{frame}

\begin{frame}{The Convolution Theorem}
\textbf{Time Domain Convolution $\leftrightarrow$ Frequency Domain Multiplication}:
\[
    y[n] = x[n] * h[n] \overset{\mathcal{F}}{\longleftrightarrow} Y(e^{j\omega}) = X(e^{j\omega})H(e^{j\omega})
\]

% \vspace{0.3cm}
\textbf{Why This Works}:
\begin{itemize}
    \item Complex exponentials are eigenfunctions of LTI systems
    \item Input $e^{j\omega n}$ produces output $H(e^{j\omega})e^{j\omega n}$
    \item By superposition, each frequency component is scaled by $H(e^{j\omega})$
\end{itemize}

% \vspace{0.3cm}
\textbf{Practical Impact}:
\begin{itemize}
    \item Convolution (complex operation) $\rightarrow$ Multiplication (simple)
    \item Foundation for frequency-domain filtering
    \item Basis for fast convolution algorithms
\end{itemize}


\end{frame}

\begin{frame}{The Modulation (Windowing) Theorem}
\textbf{Time Domain Multiplication $\leftrightarrow$ Frequency Domain Convolution}:
\[
    y[n] = x[n]w[n] \overset{\mathcal{F}}{\longleftrightarrow} Y(e^{j\omega}) = \frac{1}{2\pi}\int_{-\pi}^{\pi} X(e^{j\theta})W(e^{j(\omega-\theta)})d\theta
\]

% \vspace{0.3cm}
\textbf{Key Points}:
\begin{itemize}
    \item This is \textbf{periodic convolution} (circular convolution)
    \item Integration over one period only
    \item Dual to the convolution theorem
\end{itemize}

% \vspace{0.3cm}
\textbf{Applications}:
\begin{itemize}
    \item \textbf{Windowing}: Multiplying by window function in time domain
    \item \textbf{Modulation}: Shifting spectrum by multiplication with $e^{j\omega_0 n}$
    \item \textbf{Spectral analysis}: Effect of finite-length observation
\end{itemize}


\end{frame}

\section{Examples}


\begin{frame}{Example: Using Tables and Theorems}
\textbf{Problem}: Find DTFT of $x[n] = a^n u[n-5]$
\end{frame}


\ifthenelse{\boolean{showresults}}{
\begin{frame}{Example: Using Tables and Theorems}
\textbf{Problem}: Find DTFT of $x[n] = a^n u[n-5]$

% \vspace{0.3cm}

  \textbf{Solution}:
  \begin{enumerate}
      \item Start with known transform: $a^n u[n] \leftrightarrow \frac{1}{1-ae^{-j\omega}}$
      
      \item Apply time-shifting theorem:
      \[
          a^n u[n-5] \leftrightarrow e^{-j5\omega} \cdot \frac{1}{1-ae^{-j\omega}}
      \]
      
      \item Account for the factor $a^5$:
      \[
          x[n] = a^5 \cdot a^{n-5} u[n-5]
      \]
      
      \item Final result:
      \[
          X(e^{j\omega}) = \frac{a^5 e^{-j5\omega}}{1-ae^{-j\omega}}
      \]
  \end{enumerate}
\end{frame}
}


\begin{frame}{Example: Inverse Transform Using Partial Fractions}
\textbf{Given}: $X(e^{j\omega}) = \frac{1}{(1-ae^{-j\omega})(1-be^{-j\omega})}$
\end{frame}


\ifthenelse{\boolean{showresults}}{
  \begin{frame}{Example: Inverse Transform Using Partial Fractions}
  \textbf{Given}: $X(e^{j\omega}) = \frac{1}{(1-ae^{-j\omega})(1-be^{-j\omega})}$

  % \vspace{0.3cm}
  \textbf{Partial Fraction Expansion}:
  \[
      X(e^{j\omega}) = \frac{A}{1-ae^{-j\omega}} + \frac{B}{1-be^{-j\omega}}
  \]

  % \vspace{0.3cm}
  \textbf{Finding Constants}:
  \begin{itemize}
      \item $A = \frac{a}{a-b}$ (residue at $e^{j\omega} = a$)
      \item $B = \frac{-b}{a-b}$ (residue at $e^{j\omega} = b$)
  \end{itemize}

  % \vspace{0.3cm}
  \textbf{Inverse Transform}:
  \[
      x[n] = \frac{a}{a-b} a^n u[n] - \frac{b}{a-b} b^n u[n]
  \]

  % \vspace{0.3cm}
  \textbf{Key Technique}: Decompose complex transforms into simpler known forms
  \end{frame}
}





\begin{frame}{Example: Highpass Filter Design}
\textbf{Desired Frequency Response}:
\[
    H(e^{j\omega}) = \begin{cases}
        e^{-j\omega n_d}, & \omega_c < |\omega| < \pi \\
        0, & |\omega| < \omega_c
    \end{cases}
\]

% \vspace{0.3cm}
\textbf{Approach}: Express as $H(e^{j\omega}) = e^{-j\omega n_d}(1 - H_{lp}(e^{j\omega}))$

\end{frame}

\ifthenelse{\boolean{showresults}}{
\begin{frame}{Example: Highpass Filter Design}
\textbf{Desired Frequency Response}:
\[
    H(e^{j\omega}) = \begin{cases}
        e^{-j\omega n_d}, & \omega_c < |\omega| < \pi \\
        0, & |\omega| < \omega_c
    \end{cases}
\]

% \vspace{0.3cm}
\textbf{Approach}: Express as $H(e^{j\omega}) = e^{-j\omega n_d}(1 - H_{lp}(e^{j\omega}))$

% \vspace{0.3cm}
\textbf{Impulse Response}:
\begin{align}
    h[n] &= \delta[n-n_d] - h_{lp}[n-n_d] \\
    &= \delta[n-n_d] - \frac{\sin(\omega_c(n-n_d))}{\pi(n-n_d)}
\end{align}

% \vspace{0.3cm}
\textbf{Design Insights}:
\begin{itemize}
    \item Linear phase: $e^{-j\omega n_d}$ (constant group delay)
    \item Noncausal for $n_d < \infty$
    \item Practical implementation requires windowing/truncation
\end{itemize}
\end{frame}
}





\begin{frame}{Example: Solving Difference Equations}
\textbf{System}: $y[n] - \frac{1}{2}y[n-1] = x[n] - \frac{1}{4}x[n-1]$

% \vspace{0.3cm}
\textbf{Finding Frequency Response}:
\begin{enumerate}
    \item Apply DTFT to both sides:
    % \[
    %     Y(e^{j\omega}) - \frac{1}{2}e^{-j\omega}Y(e^{j\omega}) = X(e^{j\omega}) - \frac{1}{4}e^{-j\omega}X(e^{j\omega})
    % \]
    
    \item Factor and solve:
    % \[
    %     H(e^{j\omega}) = \frac{Y(e^{j\omega})}{X(e^{j\omega})} = \frac{1 - \frac{1}{4}e^{-j\omega}}{1 - \frac{1}{2}e^{-j\omega}}
    % \]
    
    \item Partial fractions:
    % \[
    %     H(e^{j\omega}) = \frac{1}{1 - \frac{1}{2}e^{-j\omega}} - \frac{\frac{1}{4}e^{-j\omega}}{1 - \frac{1}{2}e^{-j\omega}}
    % \]

   \item Inverse Transform:

  
\end{enumerate}
\end{frame}


\ifthenelse{\boolean{showresults}}{
  \begin{frame}{Example: Solving Difference Equations}
  \textbf{System}: $y[n] - \frac{1}{2}y[n-1] = x[n] - \frac{1}{4}x[n-1]$

  % \vspace{0.3cm}
  \textbf{Finding Frequency Response}:
  \begin{enumerate}
      \item Apply DTFT to both sides:
      \[
          Y(e^{j\omega}) - \frac{1}{2}e^{-j\omega}Y(e^{j\omega}) = X(e^{j\omega}) - \frac{1}{4}e^{-j\omega}X(e^{j\omega})
      \]
      
      \item Factor and solve:
      \[
          H(e^{j\omega}) = \frac{Y(e^{j\omega})}{X(e^{j\omega})} = \frac{1 - \frac{1}{4}e^{-j\omega}}{1 - \frac{1}{2}e^{-j\omega}}
      \]
      
      \item Partial fractions:
      \[
          H(e^{j\omega}) = \frac{1}{1 - \frac{1}{2}e^{-j\omega}} - \frac{\frac{1}{4}e^{-j\omega}}{1 - \frac{1}{2}e^{-j\omega}}
      \]
  \end{enumerate}
  \end{frame}

  \begin{frame}{Example: Impulse Response from Difference Equation}
    \textbf{Continuing from previous slide}:

    % \vspace{0.3cm}
    \textbf{Inverse Transform}:
    \begin{itemize}
        \item First term: $\left(\frac{1}{2}\right)^n u[n]$
        \item Second term uses time-shifting: $-\frac{1}{4}\left(\frac{1}{2}\right)^{n-1} u[n-1]$
    \end{itemize}

    % \vspace{0.3cm}
    \textbf{Final Result}:
    \[
        h[n] = \left(\frac{1}{2}\right)^n u[n] - \frac{1}{4}\left(\frac{1}{2}\right)^{n-1} u[n-1]
    \]

    % \vspace{0.3cm}
    \textbf{Verification}:
    \begin{itemize}
        \item $h[0] = 1 - 0 = 1$ 
        \item $h[1] = \frac{1}{2} - \frac{1}{4} = \frac{1}{4}$ 
        \item System is causal and stable ($|a| = \frac{1}{2} < 1$)
    \end{itemize}
\end{frame}
}





\section{Summary}

\begin{frame}{Summary of DTFT}
\textbf{Key Concepts}:
\begin{enumerate}
    \item \textbf{DTFT Pair}:
    \begin{itemize}
        \item Analysis: $X(e^{j\omega}) = \sum_{n=-\infty}^{\infty} x[n]e^{-j\omega n}$
        \item Synthesis: $x[n] = \frac{1}{2\pi}\int_{-\pi}^{\pi} X(e^{j\omega})e^{j\omega n}d\omega$
    \end{itemize}
    
    \item \textbf{Convergence}:
    \begin{itemize}
        \item Absolute summability $\rightarrow$ uniform convergence
        \item Square summability $\rightarrow$ mean-square convergence
    \end{itemize}
    
    \item \textbf{Symmetry Properties}:
    \begin{itemize}
        \item Real sequences have conjugate-symmetric DTFTs
        \item Even/odd decomposition in both domains
    \end{itemize}
    
    \item \textbf{Transform Theorems}:
    \begin{itemize}
        \item Convolution $\leftrightarrow$ Multiplication
        \item Time shift $\leftrightarrow$ Linear phase
        \item Differentiation, modulation, Parseval's theorem
    \end{itemize}
\end{enumerate}
\end{frame}

\begin{frame}{Important Transform Pairs}
\begin{center}
\begin{tabular}{|l|l|}
\hline
\textbf{Sequence} & \textbf{DTFT} \\
\hline
$\delta[n]$ & $1$ \\
$\delta[n-n_0]$ & $e^{-j\omega n_0}$ \\
$a^n u[n]$, $|a|<1$ & $\frac{1}{1-ae^{-j\omega}}$ \\
$e^{j\omega_0 n}$ & $2\pi\sum_r \delta(\omega-\omega_0+2\pi r)$ \\
$\cos(\omega_0 n)$ & $\pi\sum_r[\delta(\omega-\omega_0+2\pi r)+\delta(\omega+\omega_0+2\pi r)]$ \\
$\frac{\sin(\omega_c n)}{\pi n}$ & Ideal lowpass filter \\
$u[n]$ & $\frac{1}{1-e^{-j\omega}} + \pi\sum_r\delta(\omega+2\pi r)$ \\
\hline
\end{tabular}
\end{center}

% \vspace{0.3cm}
\textbf{Remember}: DTFT is periodic with period $2\pi$, so we only need to consider $\omega \in [-\pi, \pi]$
\end{frame}



\end{document}


