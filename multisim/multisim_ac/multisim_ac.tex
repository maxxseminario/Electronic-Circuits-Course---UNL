\documentclass[11pt]{article}
\usepackage[margin=1in]{geometry}
\usepackage{amsmath}
\usepackage{graphicx}
\usepackage{enumitem}
\usepackage{fancyhdr}
\usepackage{xcolor}
\usepackage{tcolorbox}
\usepackage[american]{circuitikz}

\pagestyle{fancy}
\fancyhf{}
\lhead{ECEN 222 - Electronic Circuits}
\rhead{Spring 2026}
\cfoot{\thepage}

\begin{document}

\begin{center}
    {\LARGE \textbf{Multisim Transient Analysis Tutorial}}\\
    {\LARGE \textbf{Time-Domain Analysis of Passive Circuits}}\\
    \vspace{0.3cm}
    {\large ECEN 222, Spring 2026}\\
    \vspace{0.2cm}
    {\large University of Nebraska-Lincoln}\\
    {\large Maxx Seminario}\\
    \vspace{0.5cm}
\end{center}

\section*{Introduction}

This tutorial introduces AC (alternating current) analysis in Multisim, focusing on analyzing passive circuits in the time domain. You'll learn to run a transient simulation in Multisim.

\vspace{0.3cm}

\textbf{Prerequisites:}
\begin{itemize}
    \item Completion of Multisim DC Basics tutorial
    \item Understanding of phasor notation and complex numbers
    \item Familiarity with impedance concepts: $Z_R = R$, $Z_L = j\omega L$, $Z_C = 1/(j\omega C)$
\end{itemize}

\vspace{0.3cm}

\textbf{Learning Objectives:}
\begin{itemize}
    \item Build AC circuits with sinusoidal sources
    \item Use the AC function generator to create AC signals
    \item Measure AC voltages and currents with the oscilloscope
\end{itemize}

\section{Tutorial 1: Simple RC Circuit - Transient Analysis}

We'll start with a series RC circuit driven by an AC voltage source to observe the transient response.

\subsection{Step 1: Create a New Project}

\begin{enumerate}
    \item Open Multisim
    \item Click \textbf{File} $\rightarrow$ \textbf{New} $\rightarrow$ \textbf{Blank}
    \item Save your project: \textbf{File} $\rightarrow$ \textbf{Save As...} Name it \texttt{AC\_RC\_circuit}
\end{enumerate}

\subsection{Step 2: Build the Circuit}

\textbf{Components needed:}
\begin{itemize}
    \item One AC voltage source (10V RMS at 1kHz)
    \item One resistor (1k$\Omega$)
    \item One capacitor (100nF = 0.1$\mu$F)
    \item Ground
\end{itemize}

\subsubsection{Place the AC Voltage Source}

\begin{enumerate}
    \item Click \textbf{Place Source} or press \texttt{Ctrl+W}
    \item Family: \textbf{POWER\_SOURCES}
    \item Component: \textbf{AC\_POWER}
    \item Click \textbf{OK} and place it on the workspace
    \item Double-click the AC source
    \item Set:
    \begin{itemize}
        \item \textbf{Voltage (RMS)}: \texttt{10} (10V RMS)
        \item \textbf{Frequency (F)}: \texttt{1000} or \texttt{1k} (1 kHz)
        \item \textbf{Phase}: \texttt{0} degrees
    \end{itemize}
    \item Click \textbf{OK}
\end{enumerate}

\begin{center}
\includegraphics[width=0.6\textwidth]{Figures/ac_pwr_dialog.png}
\end{center}

\begin{tcolorbox}[colback=blue!5!white,colframe=blue!75!black,title=AC Source Parameters]
\begin{itemize}
    \item \textbf{RMS voltage}: The root-mean-square voltage. For a sinusoid, $V_{RMS} = V_{peak}/\sqrt{2} = 0.707 V_{peak}$
    \item \textbf{Frequency}: In Hz (cycles per second). Angular frequency $\omega = 2\pi f$
    \item \textbf{Phase}: Initial phase shift in degrees
\end{itemize}
For a 10V RMS source at 1kHz: $v(t) = 10\sqrt{2}\cos(2\pi \cdot 1000 \cdot t)$ V (peak = 14.14V)
\end{tcolorbox}

\subsubsection{Place Resistor and Capacitor}

\begin{enumerate}
    \item Place a resistor (1k$\Omega$) to the right of the AC source
    \item Place a capacitor below the resistor
    \item Double-click the capacitor and set it to \texttt{100nF} or \texttt{100e-9}
    \item Place ground below the capacitor
\end{enumerate}

\subsection{Step 3: Wire the Circuit}

Create a series RC circuit:
\begin{enumerate}
    \item Connect the positive terminal of the AC source to the resistor
    \item Connect the resistor to the top terminal of the capacitor
    \item Connect the bottom terminal of the capacitor to ground
    \item Connect ground back to the negative terminal of the AC source
\end{enumerate}

\begin{center}
\includegraphics[width=0.7\textwidth]{Figures/rc_crkt_schem.png}
\end{center}

\subsection{Step 4: Add Oscilloscope for AC Measurement}

\begin{enumerate}
    \item Go to \textbf{Simulate} $\rightarrow$ \textbf{Instruments} $\rightarrow$ \textbf{Oscilloscope}
    \item Select the \textbf{4-Channel Oscilloscope}
    \item Place it on your workspace
    \item Connect \textbf{Channel A} (positive) to the positive terminal of the AC source (input voltage)
    \item Connect \textbf{Channel A ground} to circuit ground
    \item Connect \textbf{Channel B} (positive) to the node between R and C (capacitor voltage)
    \item Connect \textbf{Channel B ground} to circuit ground
\end{enumerate}

\begin{center}
\includegraphics[width=0.7\textwidth]{Figures/oscope_conn.png}
\end{center}

\subsection{Step 5: Configure the Oscilloscope}

\begin{enumerate}
    \item Double-click the oscilloscope
    \item Set \textbf{Time base}: 500$\mu$s/div or 0.5ms/div (to see about 2 cycles at 1kHz)
    \item Set \textbf{Channel A scale}: 5V/div
    \item Set \textbf{Channel B scale}: 5V/div
    \item Set \textbf{Trigger}:
    \begin{itemize}
        \item Source: Channel A (input signal)
        \item Edge: Rising
        \item Level: 0V
    \end{itemize}
    \item Enable both Channel A and Channel B
\end{enumerate}

\subsection{Step 6: Run the Simulation}

\begin{enumerate}
    \item Click \textbf{Run} (F5)
    \item Observe both waveforms on the oscilloscope
    \item Channel A (input): Shows peak voltage of $10\sqrt{2} \approx 14.14$V (remember: oscilloscope shows peak, source is set to 10V RMS)
    \item Channel B (capacitor): Smaller amplitude sinusoid, phase-shifted
    \item Notice the capacitor voltage \textbf{lags} the input voltage
\end{enumerate}

\begin{center}
\includegraphics[width=0.8\textwidth]{Figures/sim_oscope.png}
\end{center}

\subsection{Step 7: Measure Magnitude and Phase}

Use the oscilloscope cursors to measure:

\begin{enumerate}
    \item \textbf{Magnitude of $V_C$}: Measure the peak-to-peak voltage of Channel B and divide by 2
    \item \textbf{Phase shift}: Measure the time delay between zero-crossings of the two signals
    \begin{itemize}
        \item Place cursor T1 at a rising zero-crossing of Channel A (input)
        \item Place cursor T2 at the next rising zero-crossing of Channel B (capacitor)
        \item Time difference: $\Delta t$
        \item Phase shift: $\phi = \frac{\Delta t}{T} \times 360°$ where $T = 1/f = 1$ ms
    \end{itemize}
\end{enumerate}

\subsection{Step 8: Verify with Theory}

\textbf{Calculate the theoretical values:}

At $f = 1$ kHz, $\omega = 2\pi f = 2\pi(1000) = 6283$ rad/s

\textbf{Capacitive reactance:}
\[
X_C = \frac{1}{\omega C} = \frac{1}{6283 \times 100 \times 10^{-9}} = \frac{1}{6.283 \times 10^{-4}} \approx 1592\,\Omega
\]

\textbf{Total impedance:}
\[
\mathbf{Z}_{tot} = R + \frac{1}{j\omega C} = 1000 - j1592 = 1880\angle -57.9°\,\Omega
\]

\textbf{Capacitor voltage (using voltage divider):}
\[
\mathbf{V}_C = \mathbf{V}_s \times \frac{\mathbf{Z}_C}{\mathbf{Z}_{tot}} = 10\angle 0° \times \frac{1592\angle -90°}{1880\angle -57.9°} = 8.47\angle -32.1°\text{ V (RMS)}
\]

Your oscilloscope should show:
\begin{itemize}
    \item $|V_C| \approx 8.47\sqrt{2} \approx 12.0$ V peak (oscilloscope shows peak values)
    \item Or equivalently, 8.47V RMS if you convert the oscilloscope reading
    \item Phase lag $\approx -32°$ (capacitor voltage lags input)
\end{itemize}

\begin{tcolorbox}[colback=green!5!white,colframe=green!75!black,title=Success]
If your measurements match theory, you've successfully performed transient circuit analysis in Multisim!
\end{tcolorbox}

\section{Conclusion}

You now have the skills to perform AC transient analysis in Multisim:

\begin{itemize}
    \item Configure AC voltage sources with RMS amplitude, frequency, and phase
    \item Build series RC, RL, and RLC circuits
    \item Use oscilloscope to measure AC voltages and observe waveforms
    \item Measure magnitude and phase relationships between signals
    \item Verify theoretical impedance calculations with measurements
\end{itemize}

These skills are essential for understanding frequency-domain circuit behavior.



\end{document}
