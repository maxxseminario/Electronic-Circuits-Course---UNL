\documentclass[10pt]{article}
\usepackage[utf8]{inputenc}
\usepackage{amsmath}
\usepackage{amssymb}
\usepackage{graphicx}
\usepackage{geometry}
\usepackage{enumitem}
\usepackage{multicol}
\usepackage{tikz}
\geometry{a4paper, margin=1in}

\pagestyle{empty} % Completely remove page numbers

\usepackage{ifthen}
\newboolean{showresults}
\setboolean{showresults}{true}

\title{\vspace{-1.5cm}\Large\textbf{University of Nebraska-Lincoln}\\[0.2cm]
\Large\textbf{Digital Signal Processing: Quiz 4}} 
\date{\vspace{-0.5cm}\small\textbf{September 26, 2025}}

\begin{document}

\maketitle

\noindent\textbf{Name:} \underline{\hspace{10cm}} \hfill \textbf{Total Points: 10}

\vspace{0.5cm}

\textbf{Given:} A linear time-invariant (LTI) system has the impulse response:
$$h[n] = (0.8)^n u[n] - (0.5)^n u[n-3]$$

where $u[n]$ is the unit step function.

\vspace{0.5cm}

\begin{enumerate}
    \item \textbf{(5 points)} Determine if this system is \textbf{stable}. Show your work and justify your answer using the appropriate stability criterion for discrete-time LTI systems.
    \vspace{8cm}
    \ifthenelse{\boolean{showresults}}{
        \vspace{-6cm}
    \textbf{Solution:}
    
    For a discrete-time LTI system to be stable, the impulse response must be absolutely summable:
    $$\sum_{n=-\infty}^{\infty} |h[n]| < \infty$$
    
    For $n \geq 3$:
    $$|h[n]| = |(0.8)^n - (0.5)^n| \leq (0.8)^n + (0.5)^n$$
    
    For $0 \leq n < 3$:
    $$|h[n]| = (0.8)^n$$
    
    Since both $(0.8)^n$ and $(0.5)^n$ are geometrically convergent series, the system is \textbf{stable}.
    }{}

    \item \textbf{(5 points)} Determine if this system is \textbf{causal}. Show your work and justify your answer using the appropriate causality criterion for discrete-time LTI systems.
    \vspace{8cm}
    \ifthenelse{\boolean{showresults}}{
    \vspace{-6cm}
    \textbf{Solution:}
    
    For a discrete-time LTI system to be causal, the impulse response must satisfy:
    $$h[n] = 0 \text{ for } n < 0$$
    
    Analyzing $h[n] = (0.8)^n u[n] - (0.5)^n u[n-3]$:
    
    For $n < 0$: Both $u[n] = 0$ and $u[n-3] = 0$, so $h[n] = 0$.
    
    Since $h[n] = 0$ for all $n < 0$, the system is \textbf{causal}.
     }{}

\end{enumerate}

\end{document}