\documentclass[10pt]{article}
\usepackage[utf8]{inputenc}
\usepackage{amsmath}
\usepackage{amssymb}
\usepackage{graphicx}
\usepackage{geometry}
\usepackage{enumitem}
\usepackage{multicol}
\geometry{a4paper, margin=1in}

\pagestyle{empty} % Completely remove page numbers

\title{\vspace{-1.5cm}\Large\textbf{University of Nebraska-Lincoln}\\[0.2cm]
\Large\textbf{Digital Signal Processing: Quiz 0}} % Add institution and make title larger
\date{\vspace{-0.5cm}\small\textbf{August 25, 2025}} % Format the date professionally

\begin{document}

\maketitle

\noindent\textbf{Name:} \underline{\hspace{10cm}} \hfill \textbf{Total Points: 10}

\vspace{0.5cm}

\noindent\textbf{Instructions:}
\begin{itemize}
    \item This quiz is \textbf{closed-book and individual}.
    \item Show all work for partial credit.
    \item Write neatly and clearly.
\end{itemize}

% \vspace{1cm}

\begin{enumerate}
    \item \textbf{(3 points)} Consider the discrete-time sequence \( x[n] = 2\delta[n+3] - \delta[n+1] + 3\delta[n] + 1.5\delta[n-2] \). Sketch the sequence \( x[n] \) on the discrete-time axis.
    
    \vspace{4cm}

    \item \textbf{(3 points)} A discrete-time sinusoidal signal is given as \( x[n] = \cos\left(\frac{\pi n}{4}\right) \). Determine whether this sequence is periodic. If it is, find the period \( N \).

    \vspace{4cm}

    \item \textbf{(3 points)} Express the unit impulse sequence \( \delta[n] \) in terms of the unit step function \( u[n] \). 

    \vspace{4cm}

    \item \textbf{(1 point)} Explain the difference between a continuous-time signal and a discrete-time signal, providing one example of each.
\end{enumerate}

\end{document}