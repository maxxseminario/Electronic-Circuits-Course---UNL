\DocumentMetadata{
  pdfversion=2.0,
  pdfstandard=ua-2,
  testphase={phase-III,math,table,title}
}

\documentclass[10pt]{article}
\usepackage{geometry}
\geometry{a4paper}
\usepackage{fancyhdr}
\usepackage{lastpage}
\usepackage{extramarks}
\usepackage[usenames,dvipsnames]{color}
\usepackage{graphicx}
\usepackage{listings}
\usepackage{courier}
\usepackage{lipsum}
\usepackage{caption}
\usepackage{subcaption}
\usepackage{amsmath}
\usepackage{amssymb}
\usepackage{epstopdf}
\usepackage{placeins}
\usepackage{color} 
\usepackage{fancyvrb} 
\usepackage{setspace}
\usepackage{bookmark}
\usepackage{pdfpages}
\usepackage{enumitem}
\usepackage{tikz}
\usepackage{pgfplots}
\usepackage{hyperref}
\usepackage{circuitikz}
\usepackage{siunitx}
\usepackage{titling}

\DeclareGraphicsExtensions{.pdf,.png,.jpg}
\graphicspath{{../figs/}}

\usetikzlibrary{positioning}
\usetikzlibrary{calc}

\pgfplotsset{compat=newest} 

\setlength{\parindent}{0pt}

\singlespacing

% Margins
\topmargin=-0.45in
\evensidemargin=0in
\oddsidemargin=0in
\textwidth=6.5in
\textheight=9.0in
\headsep=0.25in

% Header and footer
\fancypagestyle{plain}{
  \fancyhf{}
  \lhead{ECEN 222:  Electronic Circuits II-CE}
  \chead{Lab 3}
  \rhead{Page \thepage\ of \pageref{LastPage}}
  \lfoot{}
  \cfoot{}
  \rfoot{Maxx Seminario, mseminario2@huskers.unl.edu}
  \renewcommand\headrulewidth{0.4pt}
  \renewcommand\footrulewidth{0.4pt}
}

\title{\textbf{\Huge Bridge Rectifiers with Shunt Voltage Regulation}\\
\large Lab 3 — ECEN 222: Electronic Circuits II-CE}
\author{
\large University of Nebraska–Lincoln\\
\large Department of Electrical and Computer Engineering\\
}
\date{}

\begin{document}
\thispagestyle{fancy}
\maketitle
\rule{\textwidth}{0.5pt}

\section{Objectives}

The primary objective of this lab is to investigate full-wave rectifier configurations and implement practical voltage regulation using Zener diodes. Upon completion of this lab, students will understand the operational differences between bridge rectifiers, design and implement shunt voltage regulators using Zener diodes, characterize voltage regulation performance under varying load conditions, and analyze load regulation, line regulation, and efficiency of practical power supply circuits.  Students will also gain experience with transformer-based power supplies and understand the practical considerations in selecting between different rectifier topologies.  Through hands-on measurements and analysis, students will connect theoretical power supply design concepts with real-world regulated power supply implementation.

\section{Pre-Lab Preparation}

Before arriving at the lab session, students are required to thoroughly prepare by reading the relevant material from the course textbook.  Specifically, review Chapter 4 (Diodes) in Sedra \& Smith, focusing on sections covering full-wave rectifier circuits and Zener diode shunt regulators. Review the concepts of load regulation (how output voltage changes with load current) and line regulation (how output voltage changes with input voltage variations). Students must also complete the pre-lab questions provided in Section \ref{sec:prelab} and come prepared with a plan for organizing and recording measurement data during the lab session. 

\section{Background Theory}

\subsection{Bridge Rectifier}

The bridge rectifier, which you studied in Lab 2, uses four diodes arranged in a bridge configuration to achieve full-wave rectification without requiring a center-tapped transformer. During the positive half-cycle of the AC input, diodes $D_1$ and $D_2$ conduct, directing current through the load in one direction. During the negative half-cycle, diodes $D_3$ and $D_4$ conduct, maintaining current flow through the load in the same direction. The key characteristic of the bridge rectifier is that two diodes are always in the conduction path, resulting in a voltage drop of approximately $2V_D \approx 1.4$ V for silicon diodes. \\

For an ideal bridge rectifier with peak input voltage $V_p$, the average DC output voltage is: 

\begin{equation}
    V_{DC} = \frac{2V_p}{\pi} \approx 0.637 V_p
    \label{eq:bridge_dc_ideal}
\end{equation}

Accounting for the two diode voltage drops: 

\begin{equation}
    V_{DC} = \frac{2(V_p - 2V_D)}{\pi}
    \label{eq:bridge_dc_real}
\end{equation}

Each diode must withstand a peak inverse voltage (PIV) equal to the peak input voltage:

\begin{equation}
    \text{PIV}_{\text{bridge}} = V_p
\end{equation}

The advantages of the bridge rectifier include no requirement for a center-tapped transformer (reducing transformer cost and size) and full utilization of the transformer secondary winding (current flows through the entire winding during both half-cycles). The disadvantages include higher voltage drop due to two diodes in series (reducing efficiency, especially at low voltages) and requiring four diodes instead of two. 

\subsection{Shunt Voltage Regulation with Zener Diodes}

The simple filtered rectifier circuits studied in Lab 2 produce a DC output voltage that varies with both input voltage changes (line variations) and load current changes (load variations). Many Electronic Circuits II-CE require a stable, regulated voltage that remains constant despite these variations. The simplest form of voltage regulation uses a Zener diode in a shunt (parallel) configuration. \\

A basic Zener shunt regulator consists of a series resistor $R_S$ (often called the dropping resistor or series resistor) connected between the unregulated DC input voltage $V_{in}$ and the load, with a Zener diode connected in parallel with the load as shown in Figure \ref{fig:shunt_reg_concept}. The Zener diode is reverse-biased and operates in its breakdown region, maintaining an approximately constant voltage $V_Z$ across the load.  \\


\begin{figure}[h]
\centering
\begin{circuitikz}[american, every node/.style={font=\small}]


    \draw (0,3) to[V, l=$V_{in}$] (0,0);
    \draw (0,3) to[R, l=$R_S$, i>^=$I_S$] (3,3) coordinate (node);

    % Zener branch 
    \draw (node) to[short] (4.0,3)
          to[zD, l_=$D_Z$, v^=$V_Z$, i>_=$I_Z$, invert] (4.0,0);

    % Load branch 
    \draw (node) to[short] (6,3)
          to[R, l_=$R_L$, v^=$V_o$, i>_=$I_L$] (6,0);

    % Return
    \draw (0,0) -- (6,0);

    % Ground reference
    \node[ground] at (3,0) {};
\end{circuitikz}
\caption{Basic Zener shunt voltage regulator circuit.}
\label{fig:shunt_reg_concept}
\end{figure}



The operation principle is straightforward.  The series resistor $R_S$ drops the difference between the input voltage and the desired output voltage. The Zener diode maintains a constant voltage $V_Z$ across the load by absorbing variations in current.  When the load current decreases, the Zener current increases to maintain constant current through $R_S$ (and thus constant voltage drop across $R_S$). When the load current increases, the Zener current decreases.  Similarly, when the input voltage increases, the Zener current increases to absorb the extra current while maintaining constant output voltage. \\

By Kirchhoff's current law: 

\begin{equation}
    I_S = I_Z + I_L
    \label{eq:kcl_reg}
\end{equation}

where $I_S$ is the current through the series resistor, $I_Z$ is the Zener current, and $I_L$ is the load current. \\

The voltage across the series resistor is: \\

\begin{equation}
    V_{R_S} = V_{in} - V_Z
\end{equation}

Therefore, the current through the series resistor is: \\

\begin{equation}
    I_S = \frac{V_{in} - V_Z}{R_S}
    \label{eq:is_calc}
\end{equation}

For proper regulation, the Zener diode must remain in breakdown, which requires: \\

\begin{equation}
    I_Z \geq I_{Z,\min}
\end{equation}

where $I_{Z,\min}$ is the minimum Zener current required to maintain breakdown (specified in the datasheet). This condition must be satisfied under all operating conditions, particularly at maximum load current.\\

The maximum Zener current occurs at minimum load (no load, or $I_L = 0$):\\

\begin{equation}
    I_{Z,\max} = I_S = \frac{V_{in} - V_Z}{R_S}
\end{equation}

The power dissipated in the Zener diode is:\\

\begin{equation}
    P_Z = V_Z \cdot I_Z
\end{equation}

This must not exceed the Zener's power rating under any operating condition (specified in the datasheet). \\

\subsection{Design of Zener Shunt Regulators}

Designing a Zener shunt regulator involves selecting appropriate values for $R_S$ and the Zener diode rating. The design process typically follows these steps:\\

\begin{enumerate}
    \item Select a Zener diode with voltage rating $V_Z$ equal to the desired output voltage.
    \item Determine the range of input voltage $V_{in,\min}$ to $V_{in,\max}$ and load current $I_{L,\min}$ to $I_{L,\max}$. 
    \item Calculate the required series resistance to maintain adequate Zener current under worst-case conditions (maximum load current and minimum input voltage).
    \item Verify that the Zener power dissipation does not exceed its rating under worst-case conditions (minimum load current and maximum input voltage).
\end{enumerate}

A common design approach is to choose $R_S$ such that the current through it (when the load is disconnected) is approximately twice the maximum expected load current. This ensures that adequate current flows through the Zener to maintain regulation even when the full load is connected.\\

\subsection{Regulation Performance Metrics}

The performance of a voltage regulator is characterized by several metrics:\\

\textbf{Load Regulation:} This measures how much the output voltage changes as the load current varies from no-load to full-load, with constant input voltage:\\

\begin{equation}
    \text{Load Regulation} = \frac{V_{o,\text{no-load}} - V_{o,\text{full-load}}}{V_{o,\text{full-load}}} \times 100\%
    \label{eq:load_reg}
\end{equation}

Ideally, load regulation should be 0\% (no change in output voltage with load).\\

\textbf{Line Regulation:} This measures how much the output voltage changes as the input voltage varies over its specified range, with constant load: \\

\begin{equation}
    \text{Line Regulation} = \frac{\Delta V_o}{\Delta V_{in}} \times 100\%
    \label{eq:line_reg}
\end{equation}

or sometimes expressed as the absolute change in output voltage per unit change in input voltage. \\

\textbf{Efficiency:} This is the ratio of power delivered to the load to the total power drawn from the input:\\

\begin{equation}
    \eta = \frac{P_o}{P_{in}} = \frac{V_o I_L}{V_{in} I_S} \times 100\%
    \label{eq:efficiency}
\end{equation}

Zener shunt regulators generally have poor efficiency, especially under light load conditions, because significant power is dissipated in both the series resistor and the Zener diode.\\

\section{Experimental Procedures}

\subsection{Part 1: Bridge Rectifier with Filter}

Begin by constructing a full-wave bridge rectifier with filter capacitor as shown in Figure \ref{fig:bridge_filtered_lab3}. This circuit should be familiar from Lab 2. Use four rectifier diodes in the bridge configuration, the 1000 $\mu$F filter capacitor, and initially use a 2.2 k$\Omega$ load resistor. Connect your AC source to provide approximately 12 V RMS at 60 Hz.  Verify the polarity of the electrolytic capacitor before applying power.\\

\begin{figure}[h]
    \centering
    \begin{circuitikz}[american, scale=1.3]
        % AC source on left
        \draw (-2,0) to[sV, l=$v_{AC}$] (0,0);
        
        % Bridge diodes in diamond configuration
        \draw (0,0) to[D, l=$D_1$, *-] (1.5,1.5);
        \draw (0,0) to[D, l_=$D_2$, invert] (1.5,-1.5);
        \draw (1.5,1.5) to[D, l=$D_3$, -*, invert] (3,0);
        \draw (1.5,-1.5) to[D, l_=$D_4$] (3,0);
        
        \draw (1.5,1.5) to[short, *-] (5,1.5);
        \draw (1.5,1.5) to[short] (4,1.5)
              to[R, l^=$R_L$, v_=$V_o$] (4,-1.5)
              to[short] (1.5,-1.5);
        \draw (5,1.5) to[C, l=$C$] (5,-1.5)
              to[short, -*] (1.5,-1.5);

        \draw (-2,0) node[ground]{};
        \draw (3,0) node[ground]{};
    \end{circuitikz}
    \caption{Bridge rectifier with filter capacitor for Part 1.}
    \label{fig:bridge_filtered_lab3}
\end{figure}

Connect the oscilloscope to observe the output voltage waveform across the load resistor. Measure and record the DC output voltage $V_o$ using the DMM, the peak output voltage, the minimum output voltage, and calculate the peak-to-peak ripple voltage $V_r$.  Measure the ripple frequency and verify it is 120 Hz (twice the AC line frequency). Also measure the AC input voltage (RMS and peak values).\\

Replace the load resistor with the 1 k$\Omega$ resistor and repeat the measurements. Then use the 4.7 k$\Omega$ resistor and repeat once more. For each load value, calculate the load current as $I_L = V_o / R_L$ and observe how the DC output voltage and ripple voltage change with load current. \\

In your lab report, present oscilloscope waveforms for at least one load condition, showing the ripple voltage clearly. Create a table or plot showing how DC output voltage and ripple voltage vary with load resistance.  Explain why the output voltage decreases and ripple increases with heavier loading (smaller resistance). This unregulated bridge rectifier will serve as the baseline for comparison with the regulated version you will build later.\\

\subsection{Part 2: Design of Zener Shunt Regulator}

You will now design a Zener shunt regulator to provide a stable output voltage from the bridge rectifier circuit. First, determine the characteristics of your Zener diode.  In the datasheet, find the Zener voltage $V_Z$, minimum Zener current $I_{Z,\min}$, and maximum power rating $P_{Z,\max}$. \\

From Part 1, you have measured the DC output voltage of your bridge rectifier under different load conditions. This will serve as the input voltage $V_{in}$ to your regulator.  Assume you want to design the regulator to supply load currents from 0 mA (no load) up to 20 mA (full load) while maintaining regulation. \\

Calculate the required series resistance $R_S$ using the following design approach:\\

\begin{enumerate}
    \item Choose a design current through the series resistor.  A common choice is to make this current equal to approximately 1.5 to 2 times the maximum load current. For $I_{L,\max}$ = 20 mA, choose $I_S$ $\approx$ 40 mA.
    
    \item Calculate the series resistance: 
    \begin{equation}
        R_S = \frac{V_{in} - V_Z}{I_S}
    \end{equation}
    where $V_{in}$ is the measured DC output from your bridge rectifier (use the value measured with 2.2 k$\Omega$ load).
    
    \item Verify that the Zener power dissipation at no load does not exceed the rating:
    \begin{equation}
        P_{Z,\text{no-load}} = V_Z \cdot I_S
    \end{equation}
    This should be less than $P_{Z,\max}$. 
    
    \item Verify that the Zener current at full load is sufficient to maintain breakdown:
    \begin{equation}
        I_{Z,\text{full-load}} = I_S - I_{L,\max}
    \end{equation}
    This should be greater than $I_{Z,\min}$.
\end{enumerate}

Show all design calculations in your lab report. If your calculated $R_S$ value does not correspond to an available resistor, select the closest value and recalculate the actual currents and power dissipations.\\

\subsection{Part 3: Implementation and Testing of Shunt Regulator}

Construct the complete regulated power supply by adding the Zener shunt regulator to your bridge rectifier from Part 1. The circuit should include the AC source, bridge rectifier, 1000 $\mu$F filter capacitor, series resistor $R_S$ (the value you calculated in Part 4), Zener diode, and provision for connecting different load resistors as shown in Figure \ref{fig:regulated_supply}.\\

\begin{figure}[h]
\centering
\begin{circuitikz}[american, every node/.style={font=\small}]

    %========================
    % Bridge rectifier
    %========================
    % AC source on left
    \draw (-2,1.5) to[sV, l=$v_{AC}$] (0,1.5);
    
    % Bridge diodes in diamond configuration
    \draw (0,1.5) to[D, l=$D_1$, *-] (1.5,3);
    \draw (0,1.5) to[D, l_=$D_2$, invert] (1.5,0);
    \draw (1.5,3) to[D, l=$D_3$, -*, invert] (3,1.5);
    \draw (1.5,0) to[D, l_=$D_4$] (3,1.5);
    
    % Filter capacitor C
    \draw (1.5,3) to[short, *-] (4,3);
    \draw (4,3) to[C, l=$C$] (4,0);
    \draw (4,0) to[short, -*] (1.5,0);
    
    % Connection to regulator (positive rail)
    \draw (4,3) to[short] (5.5,3) coordinate (vraw);
    
    % Ground rail
    \draw (-2,1.5) node[ground]{};
    \draw (3,1.5) node[ground]{};
    \draw (1.5,0) to[short] (10.5,0) coordinate (Rl_neg);

    %========================
    % Shunt regulator stage
    %========================
    % Series resistor from rectified+filtered DC to regulator node
    \draw (vraw) to[R, l=$R_S$, i>^=$I_S$] (8,3) coordinate (reg);

    % Zener diode
    \draw (reg) to[short] (8.8,3)
          to[zD, l_=$D_Z$, i>_=$I_Z$, invert] (8.8,0);

    % Load branch to ground
    \draw (reg) to[short] (10.5,3)
          to[R, l_=$R_L$, v^=$V_o$, i>_=$I_L$] (10.5,0);

\end{circuitikz}
\caption{Complete regulated power supply: bridge rectifier with filter capacitor feeding a Zener shunt regulator.}
\label{fig:regulated_supply}
\end{figure}



Pay careful attention to the polarity of the Zener diode—the cathode should be connected to the positive terminal (toward $R_S$) and the anode to ground. The Zener must be reverse-biased to operate in breakdown. \\

First, test the circuit with no load connected (remove $R_L$ or set it to a very high value). Measure the output voltage $V_o$ using the DMM. This should be approximately equal to the Zener voltage $V_Z$. Measure the current through the series resistor by measuring the voltage across it and calculating $I_S = V_{R_S}/R_S$. Since there is no load, this current is entirely flowing through the Zener diode:  $I_Z = I_S$. Verify that the Zener is not overheating—calculate $P_Z = V_Z \cdot I_Z$ and ensure it is within the power rating. \\

Now connect a 4.7 k$\Omega$ load resistor and measure the output voltage. Calculate the load current as $I_L = V_o/R_L$. The output voltage should remain very close to the no-load value, demonstrating voltage regulation. Replace the load with progressively smaller resistances (2.2 k$\Omega$, 1 k$\Omega$, 470 $\Omega$), measuring the output voltage at each value. For each load condition, also measure the voltage across the series resistor to calculate $I_S$, and calculate the Zener current as $I_Z = I_S - I_L$.\\

Continue decreasing the load resistance until one of the following occurs:  (1) the output voltage begins to drop significantly, indicating loss of regulation, or (2) you reach the minimum safe load resistance based on component ratings.  Loss of regulation occurs when the Zener current becomes too small to maintain breakdown ($I_Z < I_{Z,\min}$).\\

In your lab report, create a table showing output voltage, load current, series current, Zener current, and Zener power dissipation for each load resistance value tested. Plot output voltage versus load current.  Describe what you observe about the output voltage as load current increases. At what load current does regulation begin to fail? Explain this in terms of Zener current.  Calculate the Zener current for each load condition and verify that regulation is maintained as long as $I_Z \geq I_{Z,\min}$.\\

\subsection{Part 4: Efficiency Analysis}

Calculate the efficiency of your regulated power supply under several different load conditions.  For each load resistance value from Part 5 where regulation was maintained, calculate: \\

\begin{enumerate}
    \item Output power:  $P_o = V_o \cdot I_L = V_o^2 / R_L$
    \item Input power: $P_{in} = V_{in} \cdot I_S$, where $V_{in}$ is the unregulated DC voltage from the filter capacitor
    \item Efficiency: $\eta = (P_o / P_{in}) \times 100\%$
\end{enumerate}

Also calculate the power dissipated in each component:
\begin{itemize}
    \item Series resistor: $P_{R_S} = I_S^2 \cdot R_S$
    \item Zener diode: $P_Z = V_Z \cdot I_Z$
    \item Load: $P_L = V_o \cdot I_L$ ($P_o$)
\end{itemize}

Verify that power is conserved:  $P_{in} = P_{R_S} + P_Z + P_L$.\\

In your lab report, create a table showing the power dissipations and efficiency for each load condition. Plot efficiency versus load current. Describe how efficiency varies with load.  Explain why efficiency is poor at light loads (the Zener dissipates most of the power) and improves at heavier loads (more power goes to the load, less to the Zener). Discuss the fundamental limitation of shunt regulators. They can never achieve high efficiency because they waste power in the series resistor and Zener diode.  This motivates the development of more sophisticated regulators (series regulators and switching regulators).\\

\section{Pre-Lab Questions}
\label{sec:prelab}

Complete these questions before coming to the lab session.  Include your answers and all supporting work in your lab report.\\

\begin{enumerate}

    
    \item A bridge rectifier operates from a 12 V RMS AC source. Calculate (a) the peak input voltage, (b) the peak output voltage accounting for diode drops ($V_D = 0.7$ V), (c) the average DC output voltage, and (d) the PIV each diode must withstand. Show all work.
    
    
    \item A Zener shunt regulator has $V_{in} = 15$ V (DC), $V_Z = 5.1$ V, $R_S = 220$ $\Omega$, and $R_L = 1$ k$\Omega$. Calculate (a) the current through the series resistor $I_S$, (b) the load current $I_L$, (c) the Zener current $I_Z$, (d) the power dissipated in the Zener diode, and (e) the power dissipated in the series resistor. Show all work. 
    

\end{enumerate}

\end{document}