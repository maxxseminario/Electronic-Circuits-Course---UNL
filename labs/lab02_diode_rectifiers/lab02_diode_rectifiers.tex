\DocumentMetadata{
  pdfversion=2.0,
  pdfstandard=ua-2,
  testphase={phase-III,math,table,title}
}

\documentclass[10pt]{article}
\usepackage{geometry}
\geometry{a4paper}
\usepackage{fancyhdr}
\usepackage{lastpage}
\usepackage{extramarks}
\usepackage[usenames,dvipsnames]{color}
\usepackage{graphicx}
\usepackage{listings}
\usepackage{courier}
\usepackage{lipsum}
\usepackage{caption}
\usepackage{subcaption}
\usepackage{amsmath}
\usepackage{amssymb}
\usepackage{epstopdf}
\usepackage{placeins}
\usepackage{color} 
\usepackage{fancyvrb} 
\usepackage{setspace}
\usepackage{bookmark}
\usepackage{pdfpages}
\usepackage{enumitem}
\usepackage{tikz}
\usepackage{pgfplots}
\usepackage{hyperref}
\usepackage{circuitikz}
\usepackage{siunitx}
\usepackage{titling}

\DeclareGraphicsExtensions{.pdf,.png,.jpg}
\graphicspath{{../figs/}}

\usetikzlibrary{positioning}
\usetikzlibrary{calc}

\pgfplotsset{compat=newest} 

\setlength{\parindent}{0pt}

\singlespacing

% Margins
\topmargin=-0.45in
\evensidemargin=0in
\oddsidemargin=0in
\textwidth=6.5in
\textheight=9.0in
\headsep=0.25in

% Header and footer
\fancypagestyle{plain}{
  \fancyhf{}
  \lhead{ECEN 222:  Electronic Circuits II-CE}
  \chead{Lab 2}
  \rhead{Page \thepage\ of \pageref{LastPage}}
  \lfoot{}
  \cfoot{}
  \rfoot{Maxx Seminario, mseminario2@huskers.unl.edu}
  \renewcommand\headrulewidth{0.4pt}
  \renewcommand\footrulewidth{0.4pt}
}

\title{\textbf{\Huge Diode Rectifier Circuits}\\
\large Lab 2 — ECEN 222: Electronic Circuits II-CE}
\author{
\large University of Nebraska–Lincoln\\
\large Department of Electrical and Computer Engineering\\
}
\date{}

\begin{document}
\thispagestyle{fancy}
\maketitle
\rule{\textwidth}{0.5pt}

\section{Objectives}

The primary objective of this lab is to investigate the operation of diode rectifier circuits that convert alternating current (AC) to direct current (DC). Upon completion of this lab, students will understand the operation of half-wave and full-wave rectifier circuits, observe and measure the effects of filtering capacitors on rectifier output, characterize rectifier performance through measurements of ripple voltage and DC output levels, and analyze the relationship between load resistance and rectifier performance.  Students will also gain practical experience with AC signal measurements and understand the limitations and trade-offs in rectifier circuit design.  Through hands-on measurements and analysis, students will connect theoretical rectifier concepts with real-world power supply design.
\\

\section{Pre-Lab Preparation}

Before arriving at the lab session, students are required to thoroughly prepare by reading the relevant material from the course textbook.  Specifically, review Chapter 4 (Diodes) in Sedra \& Smith, focusing on sections covering rectifier circuits, peak detection, and the use of filtering capacitors. Review AC circuit analysis including RMS voltage, peak voltage, and the relationship between them. Additionally, review the concepts of ripple voltage, conduction angle, and peak inverse voltage (PIV) as these will be critical for understanding rectifier operation.  Students must also complete the pre-lab questions provided in Section \ref{sec:prelab} and come prepared with a plan for organizing and recording measurement data during the lab session.  Proper preparation will ensure efficient use of lab time and deeper understanding of the experimental results.
\\

\section{Background Theory}

\subsection{Rectification and DC Power Supplies}

Most electronic circuits require direct current (DC) power to operate, but electrical power distribution systems deliver alternating current (AC). A rectifier is a circuit that converts AC voltage to DC voltage, forming the essential first stage of nearly all electronic power supplies. The rectifier exploits the unidirectional current flow property of diodes to allow current to flow during one half (or both halves) of the AC cycle while blocking it during the other portions.  The resulting pulsating DC voltage can then be filtered to produce a relatively smooth DC output suitable for powering electronic circuits.\\

The quality of a rectifier circuit is characterized by several key parameters: the average (DC) output voltage, the ripple voltage (the AC component superimposed on the DC output), the peak inverse voltage (PIV) that the diodes must withstand, and the efficiency of power conversion. Understanding these parameters and the trade-offs between them is essential for practical power supply design.\\

\subsection{Half-Wave Rectifier}

The simplest rectifier circuit is the half-wave rectifier, which consists of a single diode in series with the load as shown in Figure \ref{fig:halfwave_circuit}. When the AC input voltage is positive (during the positive half-cycle), the diode is forward biased and conducts current, allowing the input voltage (minus the diode forward voltage drop) to appear across the load. When the input voltage is negative (during the negative half-cycle), the diode is reverse biased and blocks current flow, resulting in zero voltage across the load.

\begin{figure}[h]
    \centering
    \begin{circuitikz}[american]
        \draw (0,0) to[sV, l=$v_{AC}$] (0,3)
              to[D, l=$D$] (3,3)
              to[R, l=$R_L$, v=$v_o$] (3,0)
              to[short] (0,0);
        \draw (1.5,0) node[ground]{};
    \end{circuitikz}
    \caption{Half-wave rectifier circuit without filter capacitor. }
    \label{fig:halfwave_circuit}
\end{figure}

The output of an unfiltered half-wave rectifier consists of a series of positive half-sinusoids separated by intervals of zero voltage. If the input voltage is $v_i(t) = V_p \sin(\omega t)$ where $V_p$ is the peak voltage and $\omega = 2\pi f$ is the angular frequency, the average (DC) value of the output voltage for an ideal diode (zero forward voltage drop) is:

\begin{equation}
    V_{DC} = \frac{V_p}{\pi} \approx 0.318 V_p
    \label{eq:halfwave_dc}
\end{equation}

For a real silicon diode with forward voltage drop $V_D \approx 0.7$ V, the peak output voltage is reduced to $V_p - V_D$, and the average output voltage becomes:

\begin{equation}
    V_{DC} = \frac{V_p - V_D}{\pi}
    \label{eq:halfwave_dc_real}
\end{equation}

The diode must withstand a reverse voltage equal to the peak input voltage $V_p$ during the negative half-cycle. Therefore, the peak inverse voltage (PIV) rating of the diode must satisfy:

\begin{equation}
    \text{PIV} \geq V_p
\end{equation}

\subsection{Full-Wave Rectifier}

A full-wave rectifier conducts current to the load during both half-cycles of the AC input, doubling the output frequency and significantly improving the DC output and filtering characteristics compared to a half-wave rectifier.  There are two common configurations: the center-tapped transformer rectifier and the bridge rectifier.\\

\subsubsection{Bridge Rectifier}

The bridge rectifier, shown in Figure \ref{fig:bridge_circuit}, uses four diodes arranged in a bridge configuration and does not require a center-tapped transformer. During the positive half-cycle of the input voltage, diodes $D_1$ and $D_2$ conduct while $D_3$ and $D_4$ are reverse biased, allowing current to flow through the load in one direction. During the negative half-cycle, diodes $D_3$ and $D_4$ conduct while $D_1$ and $D_2$ are reverse biased, but current still flows through the load in the same direction. The result is that both half-cycles of the input contribute to the output.

\begin{figure}[h]
    \centering
    \begin{circuitikz}[american, scale=.75]
        % \draw[step=1cm, gray!30, very thin] (-2,-2) grid (8,6);
        \ctikzset{label/align=straight}
        \draw (-1,0) to[sV, l=$v_{AC}$] (-1,4);
        \draw (-1,4) -- (4,4);
        \ctikzset{label distance=-.2cm}
        \draw (2,2) to [D, l=$D_4$, *-*] (4,4) to [D, l=$D_1$, *-*] (6,2);
        \draw (2,2) to [D, l_=$D_2$, *-*] (4,0) to [D, l_=$D_3$, *-*] (6,2);
        \draw (-1,0) -- (4,0);
        \ctikzset{label distance=0}
        \draw (6,2) -- (8,2) to [R, l=$R_L$, v=$v_o$] (8,-1) -- (1, -1) -- (1,2) -- (2,2);
        \draw (-1,0) to node[ground]{} (-1,0);

    \end{circuitikz}
    \caption{Full-wave bridge rectifier circuit without filter capacitor.}
    \label{fig:bridge_circuit}
\end{figure}

For an ideal bridge rectifier (zero diode drops), the average DC output voltage is:

\begin{equation}
    V_{DC} = \frac{2V_p}{\pi} \approx 0.637 V_p
    \label{eq:fullwave_dc}
\end{equation}

For a real bridge rectifier, two diodes are always conducting in series, so the output peak voltage is reduced by two diode drops ($2V_D \approx 1.4$ V):

\begin{equation}
    V_{DC} = \frac{2(V_p - 2V_D)}{\pi}
    \label{eq:fullwave_dc_real}
\end{equation}

Each diode in the bridge must withstand a PIV equal to the peak input voltage: 

\begin{equation}
    \text{PIV} \geq V_p
\end{equation}

The bridge rectifier has the advantage of not requiring a center-tapped transformer and produces twice the output voltage compared to a center-tapped configuration for the same transformer secondary voltage.

\subsection{Filtering and Ripple Voltage}

The pulsating DC output of a rectifier is not suitable for most electronic applications, which require a smooth, constant DC voltage. A filter capacitor connected in parallel with the load can smooth the output by storing charge during the conduction periods and releasing it during the non-conduction periods. The capacitor charges to approximately the peak voltage when the diode conducts and then discharges through the load resistance when the diode is off.\\

The voltage across the capacitor exhibits a sawtooth waveform, alternately charging quickly (when the diode conducts) and discharging slowly (through the load). The variation in voltage is called the ripple voltage, denoted $V_r$ (peak-to-peak). For a capacitor $C$ supplying a load resistance $R_L$ with discharge time approximately equal to the period $T$ of the rectified waveform, the ripple voltage can be approximated as:

\begin{equation}
    V_r \approx \frac{V_p}{f R_L C}
    \label{eq:ripple}
\end{equation}

where $f$ is the frequency of the rectified output (equal to the AC line frequency for half-wave rectifiers and twice the line frequency for full-wave rectifiers). This equation assumes that the ripple is small compared to the DC voltage, which is valid for well-designed power supplies.\\

The DC output voltage with filtering is approximately:

\begin{equation}
    V_{DC} \approx V_p - \frac{V_r}{2}
    \label{eq:dc_filtered}
\end{equation}

The ripple factor, defined as the ratio of the RMS ripple voltage to the DC voltage, characterizes the quality of the DC output: 

\begin{equation}
    r = \frac{V_{r,RMS}}{V_{DC}}
\end{equation}

For a well-filtered supply, the ripple factor should be much less than 1 (typically less than 0.01 or 1\%).\\

The choice of filter capacitor involves trade-offs.  A larger capacitor reduces ripple voltage but increases cost, size, and the peak current through the diodes (since the capacitor charges quickly when the diode conducts). The capacitor must also have a voltage rating exceeding the peak input voltage with appropriate safety margin. 

\subsection{Waveform Generator}

\begin{figure}[h]
    \centering
    \begin{circuitikz}[american, scale=.75]
        \draw (0,0) to[sV, l=$v_{s}$] (0,4);
        \draw (0,4) to [R, l=$50\Omega$] (4,4);
        \draw (4,4) to (6, 4) to [R, l_=$Z_{load}$, *-*, v^=$v_o$] (6,0);
        \draw (6,0) -- (0,0);
        \draw[dashed] (-1.5,-.5) -- (-1.5,5) -- (4,5) -- (4,-.5) -- (-1.5,-.5);

    \end{circuitikz}
    \caption{Waveform Generator internal series impedance}
    \label{fig:waveformgen}
\end{figure}

A quick note should be made about the waveform generator. Within the device, there is an internal impedance of 50$\Omega$ in series with the voltage source. Figure \ref{fig:waveformgen} shows how this impedance is in series with the load, effectively creating a voltage divider. The voltage drop across this impedance will cause the output voltage to vary depending on the impedance of the output. This will be noticed while conducting the experiments in this lab.\\


\section{Experimental Procedures}

\subsection{Part 1: Half-Wave Rectifier Without Filter}
\emph{Characterize the basic half-wave rectifier circuit}\\

Begin by constructing the circuit shown in Figure \ref{fig:halfwave_circuit} on the breadboard using a single rectifier diode and the 1 k$\Omega$ load resistor.  Ensure that the diode is oriented correctly with the anode connected to the AC source and the cathode toward the load resistor.  Connect the waveform generator to provide a sinusoidal voltage with an RMS value of 5.0 V at 60 Hz. The voltage limit setting of the instrument may have to be adjusted. Make sure set the output impedance setting is specified as \emph{HiZ}.\\

Connect the oscilloscope to observe both the input AC voltage and the output voltage across the load resistor. Use DC coupling on both channels.  Adjust the oscilloscope time base to display several complete cycles of the waveform clearly. Observe that the output consists of positive half-sinusoids corresponding to the positive half-cycles of the input, with zero voltage during the negative half-cycles when the diode is reverse biased.
\\

Measure and record the peak input voltage $V_{p,in}$ and the peak output voltage $V_{p,out}$ from the oscilloscope (Referred to as "Maximum" in the oscilloscope Measure menu). Note that the output peak voltage should be approximately 0.7 V less than the input peak voltage due to the diode forward voltage drop. Use the oscilloscope's measurement functions or cursors to find the period of the waveforms and verify that the frequency is 60Hz.  Measure and record the DC (average) voltage $V_{DC,out}$ across the load resistor using the oscilloscope's "Average" measurment.  This DC voltage should correspond approximately to the theoretical value given by Equation \ref{eq:halfwave_dc_real}. Be sure to record a screen capture of the oscillosope.\\

\textbf{Report}
\begin {enumerate}
    \item Provide measurements for the peak input voltage, peak output voltage, and DC output voltage.
    \item Include oscilloscope screenshots showing both the input and output voltages. Indicate the periods when the diode is conducting and when it is off.
    \item Calculate the theoretical average DC output voltage using Equation \ref{eq:halfwave_dc_real} and compare it with the measured value. Explain any discrepancies.
    \item Discuss observations about the waveform shape and explain the physical operation of the circuit during each portion of the AC cycle.  Why does the output peak voltage differ from the input peak voltage? \\
\end {enumerate}


\subsection{Part 2: Half-Wave Rectifier With Filter Capacitor}
\emph{A filter capacitor will now be added to the half-wave rectifier to observe its smoothing effect.}\\

Modify the circuit from Part 1 by connecting a 10 $\mu$F electrolytic capacitor in parallel with the 1$k\Omega$ load resistor as shown in Figure \ref{fig:halfwave_filtered}. Pay careful attention to capacitor polarity:  the positive terminal should be connected to the cathode of the diode, and the negative terminal to ground.  Reversed polarity can cause the capacitor to fail.\\

\begin{figure}[h]
    \centering
    \begin{circuitikz}[american]
        \draw (0,0) to[sV, l=$v_{AC}$] (0,3)
              to[D, l=$D$] (3,3)
              to[short] (4.5,3);
        \draw (3,3) to[R, l=$R_L$, v=$v_o$] (3,0);
        \draw (4.5,3) to[cC, l=$C$] (4.5,0);
        \draw (0,0) to[short] (4.5,0);
        \draw (2.25,0) node[ground]{};
    \end{circuitikz}
    \caption{Half-wave rectifier with filter capacitor.}
    \label{fig:halfwave_filtered}
\end{figure}

With the oscilloscope still connected across the load, observe the output voltage waveform. A characteristic sawtooth or ripple waveform should be present instead of the half-sinusoids observed in Part 1. The voltage charges quickly to near the peak input voltage when the diode conducts, then decays exponentially as the capacitor discharges through the load resistor during the portion of the cycle when the diode is off. The diode conducts only during brief intervals when the input voltage exceeds the capacitor voltage.
\\

Use the oscilloscope's Peak-to-Peak measurement to measure and record the ripple voltage $V_{r}$. Measure and record the DC (average) output voltage and the peak input voltage. The DC voltage should be significantly higher than in Part 1, approximately $V_{DC} \approx V_{peak} - V_r/2$ as given by Equation \ref{eq:dc_filtered}.
\\

Measure the period of the ripple waveform and verify that it corresponds to the period of the AC input. For a half-wave rectifier, the ripple frequency equals the AC line frequency.
\\

Repeat this procedure using a 47 $\mu$F capacitor and then with a 100 $\mu$F capacitor.
\\

\textbf{Report}
\begin {enumerate}
    \item Present oscilloscope waveforms for all three capacitor values.
    \item Provide a table of the peak input voltage, output ripple voltage, and output average DC voltage for each capacitance.
    \item Compare the measured ripple voltages with the theoretical predictions from Equation \ref{eq:ripple}. Explain why larger capacitance produces less ripple. Describe the shape of the waveform during the charging and discharging portions of the cycle.
    \item Discuss the trade-offs involved in selecting filter capacitor values—what are the advantages and disadvantages of using very large capacitors? \\
\end {enumerate}


\subsection{Part 3: Effect of Load Resistance}
\emph{The half-wave rectifier with the filter capacitor will be used to investigate how the load resistance affects rectifier performance.}\\

Starting with the 1 k$\Omega$ load resistor from Part 2, measure and record the DC output voltage and ripple voltage. Use a 10 $\mu$F capacitor for the output capacitance. Replace the load resistor with a 10 k$\Omega$ resistor and repeat the measurements. Repeat a third time using a 100 k$\Omega$ resistor.\\

For each load resistance value, calculate the DC load current as $I_{DC} = V_{DC}/R_L$ and the ripple factor $r = V_r/(2\sqrt{3} \cdot V_{DC})$, where the factor $2\sqrt{3}$ converts peak-to-peak ripple to RMS ripple assuming a triangular waveform.\\


\textbf{Report}
\begin {enumerate}
    \item Provide a table of the calculated DC load current, measured output ripple voltage, and the calculated ripple factor for each resistance. Explain what trends are observed.
    \item Why does ripple increase and DC output voltage decrease slightly with heavier loading (smaller resistance)? Discuss the implications for power supply design.
    \item What limits the maximum current that can be drawn from a simple rectifier circuit?\\
\end {enumerate}


\subsection{Part 4: Full-Wave Bridge Rectifier Without Filter}
\emph{A full-wave bridge rectifier will be constructed and characterized}\\

Build the circuit shown in Figure \ref{fig:bridge_circuit} using four rectifier diodes arranged in a bridge configuration and the 1 k$\Omega$ load resistor. Pay careful attention to diode orientation: diodes $D_1$ and $D_2$ should have their cathodes toward the positive load terminal, while diodes $D_3$ and $D_4$ should have their anodes toward the negative load terminal.  Double-check the connections before applying power, as incorrect wiring can damage the diodes or instuments.
\\

The ground clip of the oscilloscope probe is internally connected to the same ground as the waveform generator (Earth ground from the wall receptiple). In order to measure the voltage across the resistor, one cannot simply connect the ground clip to the low side of the output (As this will effectively bypass $D_2$ and $D_4$ straight to ground). Instead, a probe must connected to each side of the resistor, with each ground clip connected to the ground node of the waveform generator. The oscilloscope Math subtraction function is then used to find the differential voltage between the probes.\\

With this in mind, properly connect the oscilloscope to observe the output voltage across the load resistor. Observe that the output now consists of full-wave rectified sinusoids. The magnitude of the AC input is present at the output during both positive and negative half-cycles. The output frequency is twice the input frequency (120 Hz for 60 Hz AC input).\\

Measure and record the peak output voltage and the DC (average) output voltage. Connect one of the probes to the input to measure the peak input voltage. Note that the peak output voltage should be approximately $1.4$ V less than the peak input voltage due to the two diode forward voltage drops in the conduction path. The DC voltage should be approximately twice that of the half-wave rectifier with the same input voltage, as given by Equation \ref{eq:fullwave_dc_real}.\\

\textbf{Report}
\begin {enumerate}
    \item Provide measurements for the peak input voltage, the peak output voltage, and the DC output voltage.
    \item Include the oscilloscope waveform showing the full-wave rectified output.
    \item Compare the DC output voltage with the theoretical prediction. Explain why two diode voltage drops affect the output (identify which diodes are conducting during each half-cycle).
    \item Discuss the advantages of full-wave rectification compared to half-wave rectification in terms of DC output voltage and output frequency.
\end {enumerate}


\subsection{Part 5: Full-Wave Bridge Rectifier With Filter}
\emph{Adding a filter capacitor to the full-wave bridge rectifier}\\

Add a 47 $\mu$F filter capacitor in parallel with the load resistor in the bridge rectifier circuit, to form a filtered full-wave rectifier as shown in Figure \ref{fig:bridge_filtered}. Again, observe correct capacitor polarity.\\ 

\begin{figure}[h]
    \centering
    \begin{circuitikz}[american, scale=.75]
        % \draw[step=1cm, gray!30, very thin] (-2,-2) grid (8,6);
        \ctikzset{label/align=straight}
        \draw (-1,0) to[sV, l=$v_{AC}$] (-1,4);
        \draw (-1,4) -- (4,4);
        \ctikzset{label distance=-.2cm}
        \draw (2,2) to [D, l=$D_4$, *-*] (4,4) to [D, l=$D_1$, *-*] (6,2);
        \draw (2,2) to [D, l_=$D_2$, *-*] (4,0) to [D, l_=$D_3$, *-*] (6,2);
        \draw (-1,0) -- (4,0);
        \ctikzset{label distance=0}
        \draw (6,2) -- (8,2) to [R, l_=$R_L$] (8,-1) -- (1, -1) -- (1,2) -- (2,2);
        \draw (8,2) -- (10,2) to [cC, l_=$C$, v^=$v_o$] (10,-1) -- (8,-1);
        \draw (-1,0) to node[ground]{} (-1,0);

    \end{circuitikz}
    \caption{Full-wave bridge rectifier with filter capacitor.}
    \label{fig:bridge_filtered}
\end{figure}

Measure the output voltage waveform on the oscilloscope. There should be a ripple similar to the half-wave case but with half the period (twice the frequency). Measure the peak input voltage, the peak output voltage, the DC output voltage, and the output ripple voltage. Compare the ripple voltage with that obtained from the half-wave rectifier using the same capacitor and load values. The full-wave rectifier should produce significantly less ripple.\\

\textbf{Report}
\begin {enumerate} %lab demo
    \item \textbf{Please record a video demonstration, or alternately show the lab TA. If recording a video, it should include your face (to verify you recorded it), the circuit functioning with the oscillcope screen, and a breif explaination of what you're doing. Submit this in the \emph{Lab02 Demo - Diode Rectifiers} assignment on Canvas.}
    \item Provide measurements for the peak input voltage, the peak output voltage, the DC output voltage, and the ripple voltage.
    \item Include the oscilloscope waveform of the output voltage.
    \item Compare the performance (DC output voltage, ripple voltage, ripple frequency) with the filtered half-wave rectifier from Part 2 using the same component values. Explain why the full-wave rectifier produces less ripple.
    \item Calculate the ripple using Equation \ref{eq:ripple} (remembering that $f = 120$ Hz for full-wave rectification of 60 Hz AC) and compare with the measured values.
    \item Discuss why full-wave rectification is preferred in practical power supply applications.\\
\end {enumerate}



\section{Pre-Lab Questions}
\label{sec:prelab}

Complete these questions before coming to the lab session.  Include the answers and all supporting work in the lab report.\\

\begin{enumerate}
    \item An AC voltage source provides 12 V RMS at 60 Hz. Calculate (a) the peak voltage, and (b) the peak-to-peak voltage.  Show all work.
    
    \item For a half-wave rectifier with a 12 V RMS AC input and a silicon diode with $V_D = 0.7$ V, calculate (a) the peak output voltage, and (b) the average (DC) output voltage. Show all calculations.
    
    \item For a full-wave bridge rectifier with the same 12 V RMS AC input and silicon diodes with $V_D = 0.7$ V each, calculate (a) the peak output voltage, and (b) the average (DC) output voltage. Show all calculations.
    
\end{enumerate}

\end{document}