\DocumentMetadata{
  pdfversion=2.0,
  pdfstandard=ua-2,
  testphase={phase-III,math,table,title}
}

\documentclass[10pt]{article}
\usepackage{geometry}
\geometry{a4paper}
\usepackage{fancyhdr}
\usepackage{lastpage}
\usepackage{extramarks}
\usepackage[usenames,dvipsnames]{color}
\usepackage{graphicx}
\usepackage{listings}
\usepackage{courier}
\usepackage{lipsum}
\usepackage{caption}
\usepackage{subcaption}
\usepackage{amsmath}
\usepackage{amssymb}
\usepackage{epstopdf}
\usepackage{placeins}
\usepackage{color} 
\usepackage{fancyvrb} 
\usepackage{setspace}
\usepackage{bookmark}
\usepackage{pdfpages}
\usepackage{enumitem}
\usepackage{tikz}
\usepackage{pgfplots}
\usepackage{hyperref}
\usepackage{circuitikz}
\usepackage{siunitx}
\usepackage{titling}
\usepackage{framed}

\DeclareGraphicsExtensions{.pdf,.png,.jpg}
\graphicspath{{../figs/}}

\usetikzlibrary{positioning}
\usetikzlibrary{calc}

\pgfplotsset{compat=newest} 

\setlength{\parindent}{0pt}

\singlespacing

% Margins
\topmargin=-0.45in
\evensidemargin=0in
\oddsidemargin=0in
\textwidth=6.5in
\textheight=9.0in
\headsep=0.25in



% Header and footer
\pagestyle{fancy}


% Ensure the plain style (used by \maketitle and some environments) matches
\fancypagestyle{plain}{
  \fancyhf{}
  \lhead{ECEN 222: Electronic Circuits II-CE}
  \chead{Lab 1}
  \rhead{Page \thepage\ of \pageref{LastPage}}
  \lfoot{}
  \cfoot{}
  \rfoot{Maxx Seminario, mseminario2@huskers.unl.edu}
  \renewcommand\headrulewidth{0.4pt}
  \renewcommand\footrulewidth{0.4pt}
}

% Fancy layout (same as above) for normal pages
\fancyhf{}
\lhead{ECEN 222: Electronic Circuits II-CE}
\chead{Lab 1}
\rhead{Page \thepage\ of \pageref{LastPage}}
\lfoot{}
\cfoot{}
\rfoot{Maxx Seminario, mseminario2@huskers.unl.edu}
\renewcommand\headrulewidth{0.4pt}
\renewcommand\footrulewidth{0.4pt}

% Avoid head height warnings
\setlength{\headheight}{14pt}



\title{\textbf{\Huge I–V Characteristics of PN Junction Diodes and Zener Diodes}\\
\large Lab 1 — ECEN 222: Electronic Circuits II-CE}
\author{
    % Maxx Seminario\\
\large University of Nebraska–Lincoln \\
\large Department of Electrical and Computer Engineering
}
\date{} % specific date


\begin{document}
\thispagestyle{fancy}
\maketitle
\rule{\textwidth}{0.5pt}


\section{Objectives}

The primary objective of this lab is to investigate the current-voltage (I-V) characteristics of semiconductor diodes under various bias conditions. Upon completion of this lab, students will understand the forward-bias and reverse-bias behavior of PN junction diodes, measure and characterize key diode parameters including forward voltage drop and reverse saturation current, and explore the voltage regulation properties of Zener diodes in their breakdown region.  Students will also distinguish between large-signal and small-signal operation of diodes by calculating the small signal resistance at different operating points.  Through hands-on measurements and analysis, students will connect theoretical diode models with real-world device behavior.

\section{Pre-Lab Preparation}

Before arriving at the lab session, students are required to thoroughly prepare by reading the relevant material from the course textbook.  Specifically, read Chapter 4 (Diodes) in Sedra \& Smith, paying special attention to sections covering the ideal diode model, terminal characteristics of junction diodes, the Shockley diode equation, and Zener diodes and voltage regulation. Additionally, review Kirchhoff's voltage and current laws, as these will be essential for analyzing the circuits used in this lab. Students must also complete the pre-lab questions provided in Section \ref{sec:prelab} and come prepared with a plan for organizing and recording measurement data during the lab session.  Proper preparation will ensure efficient use of lab time and deeper understanding of the experimental results.

\section{Background Theory}

\subsection{PN Junction Diode}

A PN junction diode is a two-terminal semiconductor device formed by joining p-type and n-type semiconductor materials. This junction creates a depletion region that controls current flow through the device. The fundamental property of a diode is its ability to allow current to flow easily in one direction (forward bias) while blocking current in the opposite direction (reverse bias), making it function as an electrical one-way valve. 

The relationship between current and voltage in a PN junction diode is described by the Shockley diode equation:

\begin{equation}
    i_D = I_S \left( e^{v_D / n V_T} - 1 \right)
    \label{eq:shockley}
\end{equation}

In this equation, $i_D$ represents the diode current, $I_S$ is the reverse saturation current (typically on the order of $10^{-14}$ to $10^{-15}$ A for silicon diodes at room temperature), $v_D$ is the voltage across the diode terminals, $n$ is the ideality factor (typically between 1 and 2, depending on the manufacturing process and operating conditions), and $V_T$ is the thermal voltage.  The thermal voltage is a fundamental parameter in semiconductor physics and is given by:

\begin{equation}
    V_T = \frac{kT}{q}
\end{equation}

where $k = 1.38 \times 10^{-23}$ J/K is Boltzmann's constant, $T$ is the absolute temperature in Kelvin, and $q = 1.60 \times 10^{-19}$ C is the elementary charge. At room temperature (approximately 300 K or 27°C), the thermal voltage is approximately 25.9 mV, though it is often approximated as 25 mV for hand calculations.

\subsubsection{Forward-Bias Operation}

When a positive voltage is applied to the anode relative to the cathode ($v_D > 0$), the diode is forward biased. In this condition, the depletion region narrows, allowing majority carriers to cross the junction and conduct current. The exponential term in Equation \ref{eq:shockley} dominates, causing the current to increase exponentially with voltage. For silicon diodes, negligible current flows until the forward voltage reaches approximately 0.5 V, after which current increases rapidly.  The voltage at which significant current begins to flow is often referred to as the "turn-on" voltage or "cut-in" voltage, typically around 0.6 to 0.7 V for silicon diodes at room temperature.  Once the diode is conducting appreciable current (several milliamperes or more), the voltage across it remains relatively constant at approximately 0.7 V, a value commonly used in circuit analysis for simplification.

\subsubsection{Reverse-Bias Operation}

When a negative voltage is applied to the anode relative to the cathode ($v_D < 0$), the diode is reverse biased.  In this condition, the depletion region widens, creating a larger barrier to current flow. The exponential term in Equation \ref{eq:shockley} becomes negligibly small, and the current approaches $i_D \approx -I_S$. This reverse saturation current $I_S$ is extremely small (typically nanoamperes or picoamperes) and represents the flow of minority carriers across the junction. For practical purposes, the diode appears as an open circuit in reverse bias.  However, if the reverse voltage exceeds a certain breakdown voltage, the diode will enter breakdown and conduct large currents in the reverse direction, potentially causing permanent damage unless the diode is specifically designed for this operation (as in the case of Zener diodes).

\subsubsection{Small Signal Resistance}

While the large-signal behavior of a diode is described by the nonlinear Shockley equation, small variations around a DC operating point can be approximated using a linear model. The small-signal or small signal resistance of the diode at a given operating point Q (the DC bias point) is defined as the slope of the I-V curve at that point: 

\begin{equation}
    r_d = \frac{dv_D}{di_D} \bigg|_{Q} = \frac{nV_T}{I_D}
    \label{eq:dynamic_r}
\end{equation}

where $I_D$ is the DC bias current at the operating point.  This equation shows that the small signal resistance decreases as the DC current increases, which is an important consideration in amplifier design and small-signal analysis. For example, at a forward current of 1 mA and assuming $n = 1$, the small signal resistance would be approximately 26 $\Omega$ at room temperature. 

\subsection{Zener Diode}

A Zener diode is a special type of PN junction diode that is specifically designed and doped to operate reliably in the reverse breakdown region. Unlike regular diodes, which may be damaged by reverse breakdown, Zener diodes are manufactured to have a well-defined and stable breakdown voltage. When reverse biased beyond this Zener voltage ($V_Z$), the diode enters breakdown and conducts current while maintaining a nearly constant voltage across its terminals. This voltage regulation property makes Zener diodes extremely useful in power supply circuits, voltage reference circuits, and overvoltage protection applications.

The breakdown mechanism in Zener diodes can occur through two different physical processes depending on the doping level and resulting breakdown voltage. For Zener voltages below approximately 5 V, the dominant mechanism is quantum mechanical tunneling (true Zener breakdown). For voltages above approximately 5 V, the dominant mechanism is avalanche multiplication.  Regardless of the mechanism, the practical behavior is similar: the diode maintains an approximately constant voltage in the breakdown region. 

In the breakdown region, the Zener diode can be modeled as an ideal voltage source $V_Z$ in series with a small resistance $r_z$ called the Zener resistance or small signal resistance. The actual voltage across the Zener diode in breakdown is given by:

\begin{equation}
    V_{Z,actual} = V_Z + r_z \cdot I_Z
\end{equation}

where $I_Z$ is the current through the Zener diode.  A good Zener diode will have a small value of $r_z$, meaning the voltage remains relatively constant over a wide range of currents. This small signal resistance can be determined experimentally by measuring the change in voltage for a change in current in the breakdown region.

When using Zener diodes, it is critical to ensure that the power dissipation does not exceed the device's rating. The power dissipated in the Zener diode is $P_Z = V_Z \times I_Z$, and exceeding the maximum power rating will cause thermal damage to the device. 

\section{Experimental Procedures}

\emph{For each part, please follow the instructions to preform the experiment. Provide the requested data and answer the given questions in the report.}

\subsection{Part 1: Forward-Bias I-V Characteristics of Silicon Diodes}

%Objective
\emph{Observe the forward-bias current-voltage characteristics of a silicon diode}\\

%Procedure
Begin by constructing the circuit shown in Figure \ref{fig:forward_circuit} on the breadboard using the small-signal silicon diode and the 1 k$\Omega$ resistor. Ensure that the diode is oriented correctly with the anode connected toward the positive supply terminal and the cathode toward ground. Set up the lab instuments to measure the diode voltage $V_D$ and current $I_D$. To measure the voltage, connect the leads of the DMM (Digital Multimeter) across the diode terminals to measure the diode voltage directly.  To measure the current, there are two options:  connect a second DMM in series with the circuit (being careful to observe the current measurement limits of the meter) or measure the voltage across the resistor and calculate the current using Ohm's law ($I_D = V_R / R$). The second method is often preferred as it avoids potential issues with the meter burden voltage. \\

\begin{figure}[h]
    \centering
    \begin{circuitikz}[american]
        \draw (0,0)
              to[V, l=$V_S$, invert] (0,3)     % flipped source polarity
              to[R, l=$R$, i=$I_D$] (3,3)
              to[D, l=$D$, v=$V_D$] (3,0)
              to[short] (0,0);
        % Ground symbol at the center bottom
        \draw (1.5,0) node[ground]{};
    \end{circuitikz}
    \caption{Forward-bias pn junction I–V characteristic measurement circuit.}
    \label{fig:forward_circuit}
\end{figure}

Start with the power supply set to 0 V, and gradually increase it to 5V in small incraments ($0.25 V$).  For each supply voltage, record the voltage across the diode $V_D$ and the current through the diode $I_D$. Use smaller incraments when the diode voltage is below $0
.8 V$. Very little current flows until the diode voltage reaches approximately 0.5 to 0.6 V, after which the current increases exponentially. \\

\begin{leftbar}
Exercise caution not to exceed a current of approximately 20 mA for silicon diodes, as excessive current can cause damage.  If the diode is heating up noticeably, reduce the current immediately.\\
\end{leftbar}

Once measurements are completed for the small-signal diode, repeat the entire procedure using the silicon rectifier diode.  Similar behavior should be observed, but there may be slight differences in the forward voltage drop characteristics between the two diode types. \\


\textbf{Report}
%data
\begin {enumerate}
    \item Plot the I-V characteristic (Voltage v. Current) for both diodes with clearly labeled axes.
\end {enumerate}
%discussion
Describe the shape of the curve and how it relates to the Shockley diode equation. Explain why very little current flows below a certain voltage threshold, but a large current flows once it is reached.

\subsection{Part 2: Small-Signal Resistance Analysis}

\emph{Using the data collected in Part 1, determine the small-signal resistance of the diode at different operating points}\\

Select two distinct operating points in the forward-bias region where the diode is conducting appreciable current.  Suggested operating points are around $V_{D1} \approx 0.65$ V and $V_{D2} \approx 0.75$ V, though the exact values will depend on the measured data. \\

For each operating point, calculate the small-signal resistance by examining the slope of the I-V curve in the vicinity of that point. Use the approximation: \\

\begin{equation}
    r_d \approx \frac{\Delta V_D}{\Delta I_D}
\end{equation}

where $\Delta V_D$ and $\Delta I_D$ represent small changes in voltage and current around the operating point.

\begin{leftbar}
To obtain accurate results, take a value on either side of the operating point, and calculate the slope between these two values. A smaller interval used (while still maintaining measurement accuracy) will produce a better approximation of the true small signal resistance. \\
\end{leftbar}

After calculating the experimental small signal resistance values, compare them with the theoretical predictions from Equation \ref{eq:dynamic_r}. Use an appropriate ideality factor, $n$, and use the measured DC current at each operating point. Calculate the percentage error between measured and theoretical values.\\

\textbf{Report}
%data
\begin {enumerate}
    \item Experimental measurements and $r_d$ calculation
    \item Theoretical $r_d$ calculation and comparison to experimental values
\end {enumerate}

Discuss observations about small signal resistance relationship to operating current.  Explain physically why the a small signal resistance decreases as the forward current increases. Discuss possible sources of error in the measurements, and how the accuracy of the small signal resistance calculations are affected. \\

\subsection{Part 3: Reverse-Bias Characteristics of Silicon Diodes}

In this section, the behavior of a silicon diode under reverse-bias conditions is investigated. Carefully remove the diode from the circuit and reverse its orientation so that the cathode is now connected toward the positive supply terminal and the anode toward ground as shown in Figure \ref{fig:reverse_circuit}.  This reverse-biases the diode.  To limit the current in case of accidental breakdown, change the resistor value to 470 $\Omega$. \\

\begin{figure}[h]
    \centering
    \begin{circuitikz}[american]
        \draw (0,0)
              to[V, l=$V_S$, invert] (0,3)     % same supply
              to[R, l=$R$, i=$I_D$] (3,3)
              to[D, l=$D$, v=$V_D$, invert] (3,0) % diode flipped (cathode at top, anode at bottom) for reverse bias
              to[short] (0,0);
        \draw (1.5,0) node[ground]{};
    \end{circuitikz}
    \caption{Reverse-bias pn junction I–V characteristic measurement circuit.}
    \label{fig:reverse_circuit}
\end{figure}

Starting from 0 V, gradually increase the supply voltage, which now applies an increasing reverse voltage across the diode.  Take measurements of the reverse voltage $V_D$ (which will be negative) and the reverse current $I_D$ at increments of approximately 1 V, up to the maximum rated reverse voltage of the diode. Consult the diode datasheet or ask the lab instructor for the maximum reverse voltage rating.  Do not exceed this voltage, as doing so may cause unintended breakdown and permanent damage to regular junction diodes. \\

The reverse current should be extremely small, likely in the range of nanoamperes or microamperes, and may be difficult to measure accurately with standard multimeters. The current should remain relatively constant the reverse voltage is increased, consistent with the saturation current $I_S$ predicted by the Shockley equation.  If the current suddenly repidly increases, the breakdown voltage is likely being exceeded; decrease the voltage immediately. \\

In your lab report, present your reverse-bias measurements and describe what you observed.  Explain why the reverse current is so much smaller than the forward current at equivalent voltage magnitudes. Discuss the physical mechanism responsible for this reverse current (minority carrier diffusion). If you were unable to measure the reverse current accurately due to instrument limitations, explain this limitation and estimate an upper bound on the reverse current based on your measurements. \\

\subsection{Part 4: Zener Diode Breakdown Characteristics}

Next to investigate is the voltage regulation properties of a Zener diode operating in the breakdown region. Construct the circuit shown in Figure \ref{fig:zener_circuit} using a Zener diode and a 100 $\Omega$ resistor. Pay careful attention to the polarity:  for the Zener to operate in breakdown, its cathode (the marked end) must be connected toward the positive terminal of the supply through the resistor, and its anode must be connected to ground. This is the reverse of normal diode operation. \\

\begin{figure}[h]
    \centering
    \begin{circuitikz}[american]
        \draw (0,0) to[V, l=$V_S$, invert] (0,3)
              to[R, l=$R$, i=$I_Z$] (3,3)
              to[zD, l=$D_Z$, v=$V_Z$, invert] (3,0)
              to[short] (0,0);
        \draw (1.5,0) node[ground]{};
    \end{circuitikz}
    \caption{Zener diode characteristic measurement circuit. The Zener diode is reverse-biased with the cathode toward the positive supply. }
    \label{fig:zener_circuit}
\end{figure}

Connect one DMM to measure the voltage across the Zener diode $V_Z$ and arrange to measure the current through the Zener $I_Z$ (either with a second meter in series or by measuring the voltage across the resistor and calculating current). Beginning at 0 V, gradually increase the supply voltage $V_S$ in increments of approximately 0.5 to 1 V. Record both $V_Z$ and $I_Z$ at each supply voltage setting. Pay special attention to the behavior as the supply voltage approaches the nominal Zener voltage. \\

When the supply voltage is below the Zener voltage, very little current flows and the voltage across the Zener approximately equals the supply voltage minus the resistor drop. Once the supply voltage exceeds the Zener voltage, the diode enters breakdown.  In breakdown, the voltage across the Zener should remain relatively constant (near its rated $V_Z$ value) even as the current increases significantly. This is the voltage regulation behavior that makes Zener diodes useful.  \\

Continue increasing the supply voltage up to approximately 10 to 12 V, taking measurements at each step. Constantly monitor the power dissipation in the Zener diode, calculated as $P_Z = V_Z \times I_Z$.  Do not exceed the power rating of the Zener diode. If it becomes noticeably warm to the touch, the power limit is likely being reached and current should not be increased any further. \\

In your lab report, plot the I-V characteristic of the Zener diode showing both the low-current region (before breakdown) and the breakdown region.  Clearly identify the Zener breakdown voltage on your plot. Describe what you observe about the voltage regulation behavior—how much does the Zener voltage change as the current varies over a wide range? Explain the physical mechanism responsible for Zener breakdown. Discuss why this constant-voltage property is useful in practical circuits. \\

\subsection{Part 5: Zener Diode Small Signal Resistance}

The small signal resistance of the Zener diode in breakdown region will be found using data from Part 4. Select two measurement points where the Zener is clearly in breakdown (i.e., where the supply voltage significantly exceeds the Zener voltage). Choose points with substantially different currents—for example, one point where $V_S \approx 7$ V and another where $V_S \approx 10$ V—to get a meaningful calculation. \\

Calculate the Zener small signal resistance using: 
\begin{equation}
    r_z = \frac{\Delta V_Z}{\Delta I_Z}
\end{equation}

where $\Delta V_Z$ is the change in Zener voltage and $\Delta I_Z$ is the change in Zener current between the two selected points. A small value of $r_z$ (typically less than 50 $\Omega$) indicates good voltage regulation, as it means the voltage changes very little with current.  A larger $r_z$ indicates poorer regulation. \\

In your lab report, calculate the Zener small signal resistance and discuss what this value tells you about the quality of voltage regulation provided by your Zener diode. If available, compare your measured $r_z$ with the datasheet specification for your particular Zener diode. Explain how the small signal resistance affects the performance of a Zener diode in a voltage regulator circuit—why is a smaller small signal resistance desirable? \\

\section{Pre-Lab Questions}
\label{sec:prelab}

Complete these questions before coming to the lab session.  Include answers and all supporting work in the lab report.

\begin{enumerate}
    \item A silicon diode has a reverse saturation current of $I_S = 10^{-14}$ A and is operating at room temperature (25°C, so $V_T \approx 25$ mV). Calculate the diode current when $V_D = 0.7$ V.  Assume an ideality factor of $n = 1$. Show all work. 
    
    \item For the circuit in Figure \ref{fig:forward_circuit}, assume $V_S = 5$ V, $R = 1$ k$\Omega$, and the diode forward voltage is 0.7 V when conducting.  Calculate (a) the voltage across the resistor, (b) the current through the diode, and (c) the power dissipated by the diode. Show all calculations.
    
    \item A diode is conducting a forward current of 10 mA at room temperature. Calculate its theoretical small-signal resistance assuming $n = 1$. Show your work.
    
    \item For the Zener circuit in Figure \ref{fig:zener_circuit}, assume $V_S = 10$ V, $R = 100$ $\Omega$, and $V_Z = 5. 1$ V (and assume this voltage remains constant in breakdown). Calculate (a) the current through the Zener diode, (b) the power dissipated by the Zener diode, and (c) determine whether this exceeds a 1 W power rating. Show all work.
    
    \item Explain in your own words the difference between static (DC) resistance and dynamic (small-signal) resistance of a diode. Write the mathematical expression for calculating each type of resistance.
    
    \item Research and briefly explain how temperature affects the forward voltage drop of a silicon diode. What is the typical temperature coefficient (in mV/°C)? How might this affect your measurements if the diode heats up during testing?
\end{enumerate}


\end{document}